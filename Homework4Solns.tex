\documentclass[a4paper,english,10pt]{article}
\usepackage{%
	amsfonts,%
	amsmath,%	
	amssymb,%
	amsthm,%
	algorithm,%
%	babel,%
	bbm,%
	etex,%
	%biblatex,%
	caption,%
	centernot,%
	color,%
	dsfont,%
	enumerate,%
	epsfig,%
	epstopdf,%
	geometry,%
	graphicx,%
	hyperref,%
	latexsym,%
	mathtools,%
	multicol,%
	pgf,%
	pgfplots,%
	pgfplotstable,%
	pgfpages,%
	proof,%
	psfrag,%
	subfigure,%	
	tikz,%
	ulem,%
	url%
}	
\usepackage[noend]{algpseudocode}
\usepackage[mathscr]{eucal}
\usepgflibrary{shapes}
\usetikzlibrary{%
  	arrows,%
	backgrounds,%
	chains,%
	decorations.pathmorphing,% /pgf/decoration/random steps | erste Graphik
	decorations.text,%
	matrix,%
  	positioning,% wg. " of "
  	fit,%
	patterns,%
  	petri,%
	plotmarks,%
  	scopes,%
	shadows,%
  	shapes.misc,% wg. rounded rectangle
  	shapes.arrows,%
	shapes.callouts,%
  	shapes%
}

\theoremstyle{plain}
\newtheorem{thm}{Theorem}[section]
\newtheorem{lem}[thm]{Lemma}
\newtheorem{prop}[thm]{Proposition}
\newtheorem{cor}[thm]{Corollary}

\theoremstyle{definition}
\newtheorem{defn}[thm]{Definition}
\newtheorem{conj}[thm]{Conjecture}
\newtheorem{exmp}[thm]{Example}
\newtheorem{assum}[thm]{Assumption}
\newtheorem{axiom}[thm]{Axiom}

\theoremstyle{remark}
\newtheorem{rem}{Remark}
\newtheorem{note}{Note}
\newtheorem{fact}{Fact}

\newcommand{\norm}[1]{\left\lVert#1\right\rVert}
\newcommand{\indep}{\!\perp\!\!\!\perp}
\DeclarePairedDelimiter\abs{\lvert}{\rvert}%
\newcommand\numberthis{\addtocounter{equation}{1}\tag{\theequation}}
\newcommand{\tr}{\operatorname{tr}}
\newcommand{\R}{\mathbb{R}}
\newcommand{\N}{\mathbb{N}}
\newcommand{\E}{\mathbb{E}}
\newcommand{\Z}{\mathbb{Z}}
\newcommand{\B}{\mathscr{B}}
\newcommand{\C}{\mathcal{C}}
\newcommand{\T}{\mathscr{T}}
\newcommand{\F}{\mathcal{F}}
\newcommand{\G}{\mathcal{G}}
%\newcommand{\ba}{\begin{align*}}
%\newcommand{\ea}{\end{align*}}
\newcommand{\expect}[1]{\mathbb{E}\left[{#1}\right]}
\newcommand{\prob}[1]{\mathbb{P}\left[{#1}\right]}
\newcommand{\probo}[1]{\mathbb{P}_0\left[{#1}\right]}
\newcommand{\probi}[1]{\mathbb{P}_1\left[{#1}\right]}
\newcommand{\given}{\; \big\vert \;} 
\newcommand{\bydef}{:=}
\newcommand{\indic}[1]{\mathbbm{1}\{#1\}}
\DeclareMathOperator*{\argmax}{arg\,max}
\renewcommand{\qedsymbol}{$\blacksquare$}
\makeatletter
\def\BState{\State\hskip-\ALG@thistlm}
\makeatother

\makeatletter
\def\th@plain{%
  \thm@notefont{}% same as heading font
  \itshape % body font
}
\def\th@definition{%
  \thm@notefont{}% same as heading font
  \normalfont % body font
}
\makeatother
\date{}
\usepackage{etex,enumitem,hyperref,tikz,pgfplots}
%opening
\title{Homework 4 solutions}
%\author{Deadline}

\begin{document}
\maketitle

Some of the solutions may have mistakes. Especially solutions for problem 10 and 11.

\begin{enumerate}
\item {Problem 6 solution}\\
a. Given that $N_k$ are independent and
 \begin{equation*}
        N_k \sim \mathbb{N}(0,\sigma^2), Y_k=N_k+\mu s_k.
 \end{equation*}
Therefore 
\begin{equation*}
  Y_k \sim \mathbb{N}(\mu s_k, \sigma^2)
\end{equation*}
and 
\begin{equation*}
 f(Y_1,\cdots, Y_n|\mu)=\frac{1}{{(2\pi\sigma^2)}^{\frac{n}{2}}} \bigg{(} \exp{\frac{-\mu^2\sum_{i=1}^{n} s_{i}^2}{2\sigma^2}}\bigg{)}
\bigg{(}\exp{\frac{-\sum_{i=1}^{n} y_{i}^2}{2\sigma^2}}\bigg{)}\bigg{(}\exp{\frac{\mu\sum_{i=1}^{n} s_{i}y_{i}}{\sigma^2}}\bigg{)}.
\end{equation*}
$T(Y)=\frac{\sum_{i=1}^{n} s_{i}y_{i}}{\sigma^2}$ is a complete sufficient statistic by
thm: [SUFFICIENCY, MINIMAL SUFFICIENCY and COMPLETENESS for EXPONENTIAL FAMILIES] done in the class.
\\

b. The MVUE for $\mu$ is
\begin{equation*}
 \phi(T)=\frac{\sum_{i=1}^{n} s_{i}y_{i}}{\sum_{i=1}^{n} s_{i}^2} \\ 
 \end{equation*}
\begin{equation*}
 \text{var}(\phi(T))=\frac{\mu^2\sigma^2}{\sum_{i=1}^{n}s_{i}^2}
\end{equation*}
\\

c. Again the MVUE for $\mu$ is
\begin{equation*}
 \phi(Y_1,\cdots,Y_n)=\frac{\sum_{i=1}^{n} s_{i}y_{i}}{\sum_{i=1}^{n} s_{i}^2} \\ 
 \end{equation*}
 the MVUE for variance is
 \begin{equation*}
 \psi(Y_1,\cdots,Y_n)=\frac{\sum_{i=1}^{n}{\bigg{(}y_i-\frac{\sum_{i=1}^{n}s_{i}y_{i}}{\sum_{i=1}^{n}s_{i}^2} s_i\bigg{)}^2}}{n-1} \\ 
 \end{equation*}

 
\item{Problem 7 solution}\\ 

We need to show that L1 norm minimizer is the median. Let
\begin{align*}
 f(a)= & \int_{\mathbb{R}}|\theta-a| \pi(\theta|x) d\theta \\
  & a \int_{-\infty}^{a} \pi(\theta|x) d\theta- \int_{-\infty}^{a} \theta \pi(\theta|x) d\theta
+\int_{a}^{\infty} \theta \pi(\theta|x) d\theta - a \int_{a}^{\infty} \pi(\theta|x) d\theta
\end{align*}
Now solving for $f'(a)=0$ gives $a$ is median. It can be shown that  $f''(a)>0$ that is median
is the L1 norm minimizer

\item{Problem 8 solution}\\ 
We need to show that for the loss function give mode is the minimizer. Let
\begin{align*}
f(a)= & \int_{\{ \theta_1, \theta_2 \cdots \}}c(a,\theta) \pi(\theta|x) d\theta \\
      & \sum_{i=1}^{\infty}c(a,\theta_i) \pi(\theta_i|x). 
\end{align*}

From the given $\Delta$ exactly one of $c(a,\theta_i)$ is nonzero and is exactly 1. Hence
$\argmax f(a) = \theta_i$ where $\pi(\theta_i|x)$ is maximum. That is $\theta_i$ is posterior mode. 

\item{Problem 9 solution}\\ 
Conditional density of $X$ given $Y$ is 
$$\mathbb{N}(\mu_{X}+(\Sigma_{XY}-\mu_{X}\mu_{Y}^T)\Sigma_{Y}^{-1}(Y-\mu_{Y}), \Sigma_{X}-\Sigma_{XY}\Sigma_{Y}^{-1}\Sigma_{XY}^{T})$$

The best MMSE estimate of $X$ given $Y$ is $\mathbb{E}[X|Y]=\mu_{X}+(\Sigma_{XY}-\mu_{X}\mu_{Y}^T)\Sigma_{Y}^{-1}(Y-\mu_{Y})$ is 
linear in $Y$

\item{Problem 10 solution}\\
 
 YULE-WALKER/ WIENER-HOFF EQUATIONS become undetermined. Many linear predictors exist. The one
 with minimum variance can be chosen.
 
\item{Problem 10 solution}\\

ORTHOGONALITY THM in the notes for vectors.

\end{enumerate}
\end{document}