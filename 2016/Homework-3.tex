\documentclass[a4paper,english,12pt]{article}
\usepackage{%
	amsfonts,%
	amsmath,%	
	amssymb,%
	amsthm,%
	algorithm,%
%	babel,%
	bbm,%
	etex,%
	%biblatex,%
	caption,%
	centernot,%
	color,%
	dsfont,%
	enumerate,%
	epsfig,%
	epstopdf,%
	geometry,%
	graphicx,%
	hyperref,%
	latexsym,%
	mathtools,%
	multicol,%
	pgf,%
	pgfplots,%
	pgfplotstable,%
	pgfpages,%
	proof,%
	psfrag,%
	subfigure,%	
	tikz,%
	ulem,%
	url%
}	
\usepackage[noend]{algpseudocode}
\usepackage[mathscr]{eucal}
\usepgflibrary{shapes}
\usetikzlibrary{%
  	arrows,%
	backgrounds,%
	chains,%
	decorations.pathmorphing,% /pgf/decoration/random steps | erste Graphik
	decorations.text,%
	matrix,%
  	positioning,% wg. " of "
  	fit,%
	patterns,%
  	petri,%
	plotmarks,%
  	scopes,%
	shadows,%
  	shapes.misc,% wg. rounded rectangle
  	shapes.arrows,%
	shapes.callouts,%
  	shapes%
}

\theoremstyle{plain}
\newtheorem{thm}{Theorem}[section]
\newtheorem{lem}[thm]{Lemma}
\newtheorem{prop}[thm]{Proposition}
\newtheorem{cor}[thm]{Corollary}

\theoremstyle{definition}
\newtheorem{defn}[thm]{Definition}
\newtheorem{conj}[thm]{Conjecture}
\newtheorem{exmp}[thm]{Example}
\newtheorem{assum}[thm]{Assumption}
\newtheorem{axiom}[thm]{Axiom}

\theoremstyle{remark}
\newtheorem{rem}{Remark}
\newtheorem{note}{Note}
\newtheorem{fact}{Fact}

\newcommand{\norm}[1]{\left\lVert#1\right\rVert}
\newcommand{\indep}{\!\perp\!\!\!\perp}
\DeclarePairedDelimiter\abs{\lvert}{\rvert}%
\newcommand\numberthis{\addtocounter{equation}{1}\tag{\theequation}}
\newcommand{\tr}{\operatorname{tr}}
\newcommand{\R}{\mathbb{R}}
\newcommand{\N}{\mathbb{N}}
\newcommand{\E}{\mathbb{E}}
\newcommand{\Z}{\mathbb{Z}}
\newcommand{\B}{\mathscr{B}}
\newcommand{\C}{\mathcal{C}}
\newcommand{\T}{\mathscr{T}}
\newcommand{\F}{\mathcal{F}}
\newcommand{\G}{\mathcal{G}}
%\newcommand{\ba}{\begin{align*}}
%\newcommand{\ea}{\end{align*}}
\newcommand{\expect}[1]{\mathbb{E}\left[{#1}\right]}
\newcommand{\prob}[1]{\mathbb{P}\left[{#1}\right]}
\newcommand{\probo}[1]{\mathbb{P}_0\left[{#1}\right]}
\newcommand{\probi}[1]{\mathbb{P}_1\left[{#1}\right]}
\newcommand{\given}{\; \big\vert \;} 
\newcommand{\bydef}{:=}
\newcommand{\indic}[1]{\mathbbm{1}\{#1\}}
\DeclareMathOperator*{\argmax}{arg\,max}
\renewcommand{\qedsymbol}{$\blacksquare$}
\makeatletter
\def\BState{\State\hskip-\ALG@thistlm}
\makeatother

\makeatletter
\def\th@plain{%
  \thm@notefont{}% same as heading font
  \itshape % body font
}
\def\th@definition{%
  \thm@notefont{}% same as heading font
  \normalfont % body font
}
\makeatother
\date{}
\usepackage{enumitem}
%opening
\title{Homework-3}
\author{}

\begin{document}
\maketitle 
\begin{enumerate}
\subsection*{Chernoff Bound}
\item \hyperlink{solution1}{Problem 1}\\
Investigate the Chernoff bound for testing between the two marginal densities
\begin{align*}
p_0(y)&=\begin{cases}
1~\hspace{5pt}\mbox{if}~0\leq y \leq 1\\
0~\hspace{5pt}\mbox{otherwise}
\end{cases}\\
\mbox{and},\hspace{40pt}&\\
p_1(y)&=\begin{cases}
2y~\hspace{5pt}\mbox{if}~0\leq y \leq 1\\
0~\hspace{10pt}\mbox{otherwise}
\end{cases}
\end{align*}
for a sequence of i.i.d observations, $Y_1,Y_2,\dots,Y_n$; i.e., compute the bounds on $P_F(\delta(\tau))$ and $P_M(\delta(\tau))$.
\subsection*{SPRTs}
\item \hyperlink{solution2}{Problem 2}\\
Consider a sequence of i.i.d Bernoulli observations, $Y_1,Y_2,\dots,$ with distribution
\begin{equation*}
P(Y_k=1)=1-P(Y_k=0)=1/3
\end{equation*}
under hypothesis $H_0$, and 
\begin{equation*}
P(Y_k=1)=1-P(Y_k=0)=2/3
\end{equation*}
under hypothesis $H_1$.
\begin{enumerate}
\item Use Wald's approximation to suggest values of $A$ and $B$ so that the SPRT $(A,B)$ has maximum error probability $p^*=\max(P_F,P_M)$ approximately equal to $0.01$. Describe the resulting test in detail. Also, using Wald's approximations, give an approximation for the expected sample sizes $E\{N|H_0\}$ and $E\{N|H_1\}$.
\item Find an integer $n$ as small as you can so that the maximum error probability for the optimal test with {\em fixed} sample size $n$ is no more than $0.01$. Compare $n$ to the expected sample sizes found in part (a) (Note: you can use a Chernoff bound estimate to find $n$, rather than finding the actual smallest possible $n$).
\item Compute $p^*,~E\{N|H_0\},$ and $E\{N|H_1\}$ exactly for the test you found in part (a), and compare with the approximate values you found in part (a). [Hint: use the fact that SPRT you found in part (a) is equivalent to SPRT ($A',B'$) where $A'$ and $B'$ are integer powers of $2$.]
\end{enumerate}
\item \hyperlink{solution3}{Problem 3}\\
Let $\underline{\phi}=\{\phi_j;j=0,1,\dots,\}$ be a stopping rule ($\phi_j:\mathcal{R}^j\rightarrow \{0,1\}$) and $\underline{\delta}=\{\delta_j;j=0,1,\dots,\}$ be a terminal decision rule ($\phi_j:\mathcal{R}^j\rightarrow \{0,1\}$), $\delta_j$ being a decision rule on ($\mathcal{R}^j,\mathcal{B}^j$) for each $j \geq 0$.
\begin{equation*}
\phi_j(y_1,\dots,y_j)=\begin{cases}
0~\hspace{5pt}\mbox{if}~j\neq n\\
1~\hspace{5pt}\mbox{if}~j= n
\end{cases}
\end{equation*}
\begin{equation*}
\delta_j(y_1,\dots,y_j)=\begin{cases}
\delta(y_1,\dots,y_n)~\hspace{5pt}\mbox{if}~j= n\\
\mbox{arbitrary}~\hspace{20pt}\mbox{if}~j\neq n
\end{cases}
\end{equation*}
And, 
\begin{equation*}
r(\underline{\phi},\underline{\delta})=(1-\pi_1)R_0(\underline{\phi},\underline{\delta})+\pi_1 R_1(\underline{\phi},\underline{\delta})
\end{equation*}
where,
\begin{equation*}
R_0(\underline{\phi},\underline{\delta})=\mathbb{E}_0\{\delta_N(Y_1,\dots,Y_N)\}+C\mathbb{E}_0\{N\},
\end{equation*}
\begin{equation*}
R_1(\underline{\phi},\underline{\delta})=1-\mathbb{E}_1\{\delta_N(Y_1,\dots,Y_N)\}+C\mathbb{E}_1\{N\}.
\end{equation*}
Show that 
\begin{equation*}
V^*(\pi_1) \triangleq \underset{\underset{\phi_0=0}{\underline{\phi},\underline{\delta}}}{\min}~r(\underline{\phi},\underline{\delta}),\hspace{20pt}0\leq \pi_1 \leq 1
\end{equation*}
is concave, and equal to $C$ at $\pi_1=0$ and $\pi_1=1$.
\subsection*{Estimation}
\item \hyperlink{solution4}{Problem 4}\\
Let $X_1,\dots,X_n$ be a random sample from the pdf,
\begin{equation*}
f(x|\theta)=\theta x^{-2},\hspace{20pt}0<\theta\leq x < \infty.
\end{equation*}
\begin{enumerate}
\item What is a sufficient statistic for $\theta$?
\item Find the MLE of $\theta$.
\item Find the method of moments estimator of $\theta$.
\end{enumerate}
\item \hyperlink{solution5}{Problem 5}\\
{\bf Gaussian conjugate priors: }
\begin{enumerate}
\item Show that the class of normal distributions is a conjugate prior family for the class of normal distributions with known variance, say variance 1.
\item Show that a normal-inverse gamma distributions is conjugate prior for the class of normal distributions.
[Hint: The Normal inverse-gamma distribution has the form: \\ $f(\mu,\sigma^2|\mu_0,\lambda,\alpha,\beta)=\frac{\sqrt{\gamma}}{\sigma\sqrt{2\pi}} \frac{\beta^\alpha}{\Gamma(\alpha)} \left(\frac{1}{\sigma^2}\right)^{\alpha+1} \exp\left(-\frac{2\beta+\lambda(\mu-\mu_0)^2}{2\sigma^2} \right),~\mu \in \mathcal{R},~\sigma^2 \in \mathcal{R}^+.$]
\end{enumerate}
\item \hyperlink{solution6}{Problem 6}\\
Consider a Bernoulli random variable $X$ with distribution,
\begin{equation*}
P(X=1)=1-P(X=0)=p.
\end{equation*}
Find the maximum likelihood estimate (MLE) for $p$, given an iid random sample $X_1,X_2,\dots,X_n$.
\item \hyperlink{solution7}{Problem 7}\\
Let $X_1,X_2,\dots,X_n$ be an iid random sample drawn from a Poisson distribution with parameter $\lambda$, $X_i\sim Poi(\lambda)$. Consider two unbiased estimators $\overline{X}=\frac{1}{n}\sum\limits_{i=1}^{n} x_i$, and $S^2=\frac{1}{n-1} \sum\limits_{i=1}^{n} x_i^2$. Which one has lower Mean-Square-Error (MSE)?
\item \hyperlink{solution8}{Problem 8}\\
\begin{enumerate}
\item Let $X_1,X_2,\dots,X_n~\in\mathcal{R}$ be an iid random sample drawn from a Normal distribution with known variance $\sigma^2$, $X_i\sim \mathcal{N}(\mu,\sigma^2)$. Compute the Cramer-Rao Lower Bound (CRLB) for the variance of an unbiased estimator of mean $\mu$.
\item Repeat part (a) for $X_1,X_2,\dots,X_n~\in\mathcal{R}^d$, where $X_i\sim \mathcal{N}(\boldsymbol{\mu},\boldsymbol{\Sigma})$ iid, $\boldsymbol{\Sigma}\in \mathcal{R}^{d\times d}$ is the covariance matrix.
\end{enumerate}
\item \hyperlink{solution9}{Problem 9}\\
Consider the noisy signal model,
\begin{equation}
Y=X+N,
\end{equation}
where $N$ is a complex Gaussian random variable with mean zero and variance $\sigma^2$ (iid real and imaginary parts). Let $X=A\exp(j\phi)$ is complex valued, with constant amplitude $A$ and random phase $\phi$ distributed as uniform in $[0~2\pi]$. Derive the Maximum Likelihood estimate for the amplitude $A=|X|$. [Hint: Use the approximation $I_0(x)\approx \frac{e^x}{\sqrt{2\pi x}}$ wherever necessary, $I_0(x)=\frac{1}{2\pi} \int\limits_{0}^{2\pi} \exp[x\cos(\tau)]~d\tau$ is Bessel function of zero'th order].
\end{enumerate}
\newpage
\par{\centering\Large {Solutions}\par}
\hypertarget{solution1}{\subsection*{Solution 1}}
Given,
\begin{align*}
p_0(y)&=\begin{cases}
1~\hspace{5pt}\mbox{if}~0\leq y \leq 1\\
0~\hspace{5pt}\mbox{otherwise}
\end{cases}\\
\mbox{and},\hspace{40pt}&\\
p_1(y)&=\begin{cases}
2y~\hspace{5pt}\mbox{if}~0\leq y \leq 1\\
0~\hspace{10pt}\mbox{otherwise}
\end{cases}
\end{align*}
for a sequence of i.i.d observations, $Y_1,Y_2,\dots,Y_n$. We have,
\begin{align*}
P_F(\delta)&\leq\exp(-s\tau+\mu_{T,0}(s)),~s>0,\\
P_M(\delta)&\leq\exp(\tau)\exp(-s\tau+\mu_{T,0}(s)), s\leq 1.
\end{align*} 
where, $\mu_{T,0}(s)=\log \mathbb{E}_0[e^{(s\log L(y)}]=\log \mathbb{E}_0[L(y)^{s}]$. The Likelihood Ratio is,
\begin{align*}
L(y)&=\prod_{i=1}^{n}\frac{p_1(y_{i})}{p_0(y_{i})},\\
&=\prod_{i=1}^{n} 2y_{i},\\
&=2^{n}\prod_{i=1}^{n} y_{i}.
\end{align*}
We get,
\begin{equation*}
E_0[L(y)^{s}] = 2^{ns} (E_{0}[y_{1}^{s}])^{n}= 2^{ns}(\int_{0}^{1}y^{s}ds)^{n}= \frac{2^{ns}}{(s + 1)^n}.
\end{equation*}
Hence, $\mu_{T,0}(s) = ns\log2 - n\log(s + 1)$.
\begin{enumerate}
\item[Case 1:] $\;\tau<n\log2 $.
In this case, $P_F(\delta)\leq 1 $, and $P_M(\delta)\leq \exp(\tau) $.
\item[Case 2:] $\;\tau>n\log2  $.
Here, $P_F(\delta)\leq \exp(-\tau)$, and $P_M(\delta)\leq 1 $.
\end{enumerate}

\hypertarget{solution2}{\subsection*{Solution 2}}
Given $Y_1,Y_2,.....$ are iid Bernoulli observations with distribution
\begin{equation*}
P(Y_k=1)=1-P(Y_k=0)=1/3
\end{equation*}
under hypothesis $H_0$, and 
\begin{equation*}
P(Y_k=1)=1-P(Y_k=0)=2/3
\end{equation*}
under hypothesis $H_1$.
\begin{enumerate}[label=(\alph*).]
\item From Wald's approximation,
\begin{equation*}
A \cong\frac{\gamma}{1-\alpha},~B\cong\frac{1-\gamma}{\alpha}.
\end{equation*}
Substituting for $\alpha=\gamma=0.01$, we get
\begin{equation*}
A\leq\frac{1}{99},~B\geq99.
\end{equation*}
The likelihood ratio is given by,
\begin{equation*}
L(Y_k)= \log\frac{P(Y_k|\;H_1)}{P(Y_k|\;H_0)} =
\begin{cases}
\log2~~~~~~~~Y_k=1 \\
-\log2~~~~~Y_k=0
\end{cases}
\end{equation*}
The SPRT(Sequential Probability Ratio Test) compares the cumulative likelihood ratio test to an upper and lower threshold. It declares one of the hypothesis when the hypothesis crosses the corresponding threshold.
\item Consider hypothesis $\;H_1$, using Chernoff bound we get,
\begin{equation*}
\gamma \simeq\;\exp(-2n(P-\frac{1}{2})^{2})
\end{equation*}
where $P=P(Y_k=1|\;H_1)=\frac{2}{3}$. Hence $0.01 = \;\exp(-\frac{n}{18})$ giving $n = 36 \ln 10\approx 83 \;$samples.
\item $P^{*}=max(P_{F},P_{M})=0.01$. If we choose $A\leq\frac{1}{99}$ and $B\geq99,$ we can always have $P^{*}=0.01$.
\begin{align*}
E[N| H_0]&= (\frac{1}{\mu_0})[(1-\alpha)\log(\frac{\gamma}{1-\alpha})+\alpha\;\log(\frac{1-\gamma}{\alpha})],\\
E[N| H_1]&= (\frac{1}{\mu_1})[\gamma\;\log(\frac{\gamma}{1-\alpha})+(1-\gamma)\log(\frac{1-\gamma}{\alpha})].
\end{align*}
where $\mu_{j}=\mathbb{E}\left[\log\left(\frac{P_{1}(Y_{1})}{P_{0}(Y_{1})}\right)|\;H_{j}\right]$.
\begin{align*}
\mu_{0}=\frac{1}{3}\log\left(\frac{\frac{2}{3}}{\frac{1}{3}}\right) +\frac{2}{3}\log\left(\frac{\frac{1}{3}}{\frac{2}{3}}\right)=\frac{1}{3}\log2 - \frac{2}{3} \log2=- \frac{1}{3} \log2.
\end{align*}
\begin{align*}
\mu_{1}=\frac{2}{3}\log\left(\frac{\frac{2}{3}}{\frac{1}{3}}\right) +\frac{1}{3}\log\left(\frac{\frac{1}{3}}{\frac{2}{3}}\right)=\frac{2}{3}\log2 - \frac{1}{3} \log2=\frac{1}{3} \log2.
\end{align*}
\end{enumerate}
Choose$ \;A=\frac{1}{99},~B=99$ gives $\alpha=\gamma=0.01$. Then 
\begin{equation*}
\mathbb{E}[N| H_0]= \frac{-3}{\log2}[0.99\log(\frac{0.01}{0.99})+0.01\log(\frac{0.99}{0.01})]=19.49,
\end{equation*}
and
\begin{equation*}
\mathbb{E}[N| H_1]= \frac{3}{log2}[0.01\log(\frac{0.01}{0.99})+0.99
\log(\frac{0.99}{0.01})]=19.49.
\end{equation*}
The calculated values of $E[N| H_0]$ and $E[N| H_1]$ will vary depending on the exact $\alpha$,$\gamma$ we choose and in effect the $A$ and $B$ chosen.
\hypertarget{solution3}{\subsection*{Solution 3}}
Let $\bar{\phi}$ = $\{\phi_{j};j=0,1,2,...\}$ be a stopping rule and, $\bar{\delta}=\{\delta_{j};j=0,1,2,...\}$ be a terminal decision rule. A sequential decision rule is pair of sequences $(\bar{\phi},\bar{\delta})$, the rule $(\bar{\phi},\bar{\delta})$ makes the  decision $\delta_{N}(y_{1},y_{2},...y_{N})$, where $N$ is the stopping time defined as,
\begin{equation}
N=\min\{n| \phi_{n}(y_{1},y_{2},...y_{n})=1\}
\end{equation}
The sequential decision rule of $n$-samples is defined by,
\begin{equation*}
\phi_j(y_1,\dots,y_j)=\begin{cases}
0~\hspace{5pt}\mbox{if}~j\neq n\\
1~\hspace{5pt}\mbox{if}~j= n
\end{cases}
\end{equation*}
\begin{equation*}
\delta_j(y_1,\dots,y_j)=\begin{cases}
\delta(y_1,\dots,y_n)~\hspace{5pt}\mbox{if}~j= n\\
\mbox{arbitrary}~\hspace{20pt}\mbox{if}~j\neq n
\end{cases}
\end{equation*}
The priors $ \pi_{1}$ and $ \pi_{0}$ are assigned to hypothesis $ H_{1} $ and $H_{0},$ respectively. The Bayes risk is given by,
\begin{equation*}
r(\bar{\phi},\bar{\delta})=(1-\pi_1)R_0(\bar{\phi},\bar{\delta})+\pi_1 R_1(\bar{\phi},\bar{\delta})
\end{equation*}
Consider the function,
\begin{equation*}
V^*(\pi_1) \triangleq \underset{\underset{\phi_0=0}{\underline{\phi},\underline{\delta}}}{\min}~r(\underline{\phi},\underline{\delta}),\hspace{20pt}0\leq \pi_1 \leq 1.
\end{equation*}
The relationship between Bayes risk and uniform costs has been divided into two decisions rules.
The one that takes no samples and decides $H_{1}$, (i.e., $\phi_{0}=\delta_{0}=1$) and the one that takes no samples and decides $H_{0}$, (i.e., $\phi_{0}=1-\delta_{0}=1$).
\begin{equation*}
r(\bar{\phi},\bar{\delta})|_{\phi_{0}=\delta_{0}=1}=(1-\pi_{1}),
\end{equation*}
and,
\begin{equation*}
r(\bar{\phi},\bar{\delta})|_{\phi_{0}=1-\delta_{0}=1}=\pi_{1}.
\end{equation*}
%\begin{figure}[ht!]
%\centering
%\includegraphics[width=90mm]{1.png}
%\caption{Relationships yielding the Bayes sequential rule for uniform
%costs of errors and cost С per sample. \label{overflow}}
%\end{figure}
By inspection of the figure, we see that Bayes rule for a fixed prior $\pi_{1}$ is,
\begin{equation*}
\phi_{0}=
\begin{cases}
1-\delta_{0} = 1~~~~\pi_{1}\leq\pi_{L} \\
\delta_{0} = 1~~~~~~~\pi_{L}\geq\pi_{U}
\end{cases}
\end{equation*}
and it is the decision rule with the minimum Bayes risk among all $(\bar \phi,\bar \delta)$ with $\phi_{0}$ if $\pi_{L}<\pi_{1}<\pi_{U} $. So if $\pi_{1}\leq\pi_{L}$ we take no samples and choose $H_{0}$, if $\pi_{1}\geq \pi_{U}$ we take no samples and choose $H_{1}$. Otherwise we take at-least one sample.
\hypertarget{solution4}{\subsection*{Solution 4}}
Let $X_1,X_2,....X_n$ be a random sample, $f(x|\theta)=\theta\;x^{-2} \; ;\;0<\theta\leq\;x<\infty$. The maximum likelihood estimator of $\theta$ maximizes $L_\theta(x) = \prod_{i=1}^{n} f(x_i|\theta)$.
\begin{enumerate}[label=(\alph*).]
\item 
\begin{align*}
f(\bar{x}|\theta)&= \prod_{i=1}^{n} f(x_i|\theta),\\
&=\prod_{i=1}^{n} \theta\;x_{i}^{-2} I_{[\theta,\infty]}(x_{i}),\\
&=(\prod_{i=1}^{n} \;x_{i}^{-2})\theta^{n}I_{[\theta,\infty]}(x_{(1)}) \hspace{1cm} ( x_{(1)} \mbox{ is minimum of } X).
\end{align*}
Hence by factorization theorem $X_{(1)}$ is a sufficient statistic for $\theta$.
\item 
\begin{equation*}
L(\theta|\bar{x}) = \theta^{n}(\prod_{i=1}^{n} \;x_{i}^{-2})I_{[\theta,\infty]}(x_{(1)})
\end{equation*}
$\theta^{n}$ is increasing in $\theta$. The second term of above equation doesn't depend on $\theta$. So to maximize the MLE $L(\theta|\bar x)$, we want to make $\theta$ as large as possible. But because of the indicator function,
\begin{equation*}
L(\theta|\bar x) = 0 ;~\theta>x_{(1)}.
\end{equation*}
Thus, $\hat{\theta} = x_{(1)}$.
\item $E[X] = \int_\theta^{\infty}\theta x^{-1}\,dx = \infty $. Hence, the method of moments estimator of $ \theta $  doesn't exit.
\end{enumerate}
\hypertarget{solution5}{\subsection*{Solution 5}}
\begin{enumerate}[label=(\alph*).]
\item If the posterior distributions are in the same family as the prior probability distribution, the prior and posterior are then called conjugate distributions, and the prior is called a conjugate prior for the likelihood function. By Bayes' rule $P(\mu | x) = \frac{P(x| \mu)P(\mu)}{P(x)}$. Since $P(x)$ is independent of $\mu$ the above equation can be written as,
\begin{equation*}
P(\mu | x) \propto P(x| \mu)P(\mu).
\end{equation*}
i.e., 
\begin{equation*}
\mbox{posterior distribution} \propto \mbox{Likelihood} \times \mbox{Prior distribution}.
\end{equation*}
$X_1,X_2,....,X_n$ is an iid random sample with $X_i \sim N(\mu,\sigma_{x}^{2})$. The likelihood function is,
\begin{equation*}
P(X | \mu) = \prod_{i=1}^{n} \frac{1}{\sqrt{2\pi \sigma^{2}}} \;e^{-\frac{(x_{i}-\mu)^{2}}{2\sigma_{x}^{2}}},
\end{equation*}
and the prior,
\begin{equation*}
P(\mu) = \frac{1}{\sqrt{2\pi \sigma^{2}}} e^{-\frac{(\mu-\mu_{0})^{2}}{2\sigma_{\mu}^{2}}}.
\end{equation*}
Now,
\begin{align*}
\mbox{Likelihood} \times \mbox{Prior} &= \prod_{i=1}^{n} \frac{1}{\sqrt{2\pi \sigma^{2}}} \;e^{-\frac{(x_{i}-\mu)^{2}}{2\sigma_{x}^{2}}} \times \frac{1}{\sqrt{2\pi \sigma^{2}}} e^{\frac{(\mu-\mu_{0})^{2}}{2\sigma_{\mu}^{2}}},\\
P(\mu | x)&= \frac{1}{(\sqrt{2\pi \sigma^{2}})^{n}} \;e^{-\frac{\sum_{i=1}^{n}(x_{i}-\mu)^{2}}{2\sigma_{x}^{2}}} \times \frac{1}{\sqrt{2\pi \sigma^{2}}} e^{-\frac{(\mu-\mu_{0})^{2}}{2\sigma_{\mu}^{2}}},\\
&\propto \;e^{-\frac{\sum_{i=1}^{n}(x_{i}-\mu)^{2}}{2\sigma_{x}^{2}}} \times e^{-\frac{(\mu-\mu_{0})^{2}}{2\sigma_{\mu}^{2}}},\\
&=\exp\left[{\frac{-\sum_{i=1}^{n}x_{i}^{2}+2\mu n \bar X-n\mu^{2}}{2\sigma_{x}^{2}}}-\frac{\mu^{2} - 2\mu \mu_{0}+\mu_{0^{2}}}{2\sigma_{\mu}^{2}}\right].
\end{align*}
We are interested only in distribution of $\mu$, neglecting the terms not having $\mu$, we get
\begin{align*}
P(\mu | x)&\propto \exp\left[\frac{2\mu n\bar X -n\mu^{2}}{2\sigma_{x}^{2}} - \frac{\mu^{2} - 2\mu \mu_{0}}{2\sigma_{\mu}^{2}}\right],\\
&\propto \exp\left[-\frac{\mu^{2}}{2}(\frac{1}{\sigma_{\mu}^{2}}+\frac{n}{\sigma_{x}^{2}}) - \mu(\frac{\mu_{0}}{\sigma_{\mu}^{2}}+\frac{n\bar X}{\sigma_{x}^{2}}\right]).
\end{align*}
Let, $\sigma_{\mu'}^{^{2}} = [\frac{1}{\sigma_{\mu}^{2}}+\frac{n}{\sigma_{x}^{2}}]^{-1}$, $\mu' = \sigma_{\mu'}^{2}[\frac{\mu_{0}}{\sigma_{\mu}^{2}}+\frac{n\bar X}{\sigma_{x}^{2}}]$. Substituting $\sigma_{\mu'}^{^{2}}, \mu'$ in above equations, we get
\begin{equation*}
P(\mu | x)\propto \exp\left[-\frac{\mu^{2}}{2\sigma_{\mu'}^{2}}+\frac{\mu \mu'}{\sigma_{\mu'}^{2}}\right].
\end{equation*}
Multiply the above term by a constant,
\begin{align*}
P(\mu | x)&\propto \exp\left[-\frac{\mu^{2}}{2\sigma_{\mu}^{'^{2}}}+\frac{\mu \mu^{'}}{\sigma_{\mu}^{'^{2}}}-\frac{\mu^{'^{2}}}{2\sigma_{\mu}^{'^{2}}}\right],\\
&= \exp\left[-\frac{(\mu-\mu^{'})^{2}}{2\sigma_{\mu}^{2}}\right].
\end{align*}
Normalizing the above equation, so that it is equal to one when integrated over sample space, we get,
\begin{equation*}
f(\mu|X)= \frac{1}{\sqrt{2\pi \sigma_{\mu}^{'^{2}}}} e^{-\frac{(\mu-\mu^{'})^{2}}{2\sigma_{\mu}^{2}}}.
\end{equation*}
which is the posterior distribution that is in the same family as prior distribution. Therefore the class of normal distributions is a conjugate prior family for the class of normal distributions with known variance.
\item The likelihood function a normally distributed random sample is,
\begin{equation*}
f(X | \mu) = \prod_{i=1}^{n} \frac{1}{\sqrt{2\pi \sigma^{2}}} \;e^{-\frac{(x_{i}-\mu)^{2}}{2\sigma^{2}}}.
\end{equation*}
The normal inverse-gamma distribution has the form,
\begin{equation*}
f(\mu, \sigma^{2}| \mu_{0}, \lambda,\alpha,\beta) =  \frac{\sqrt{\gamma}}{\sigma\sqrt{2\pi}} \frac{\beta^{\alpha}}{\Gamma(\alpha)} (\frac{1}{\sigma^{2}})^{\alpha + 1} \exp(-\frac{2\beta + \lambda(\mu - \mu_{0})^{2}}{2\sigma^{2}}).
\end{equation*}
Now,
\begin{align*}
f(\mu, \sigma^{2}| X)&=\frac{1}{(\sqrt{2\pi \sigma^{2}})^{n}} \;\exp\{-\frac{\sum_{i=1}^{n}(x_{i}-\mu)^{2}}{2\sigma^{2}}\} \times \frac{\sqrt{\gamma}}{\sigma\sqrt{2\pi}} \frac{\beta^{\alpha}}{\Gamma(\alpha)} (\frac{1}{\sigma^{2}})^{\alpha + 1} \exp\{-\frac{2\beta + \lambda(\mu - \mu_{0})^{2}}{2\sigma^{2}}\},\\
&\propto (\frac{1}{\sigma^{2}})^{\alpha + \frac{n}{2} + 1} \exp\{-\frac{2\beta + \sum_{i=1}^{n}(x_{i}-\mu)^{2} + \lambda(\mu - \mu_{0})^{2}}{2\sigma^{2}}\},\\
&= (\frac{1}{\sigma^{2}})^{\alpha + \frac{n}{2} + 1} \exp\{-\frac{2\beta + \sum_{i=1}^{n}x_{i}^{2}- 2\mu\sum_{i=1}^{n}x_{i}+ n\mu^{2} + \lambda(\mu^{2} - 2\mu\mu_{0} + \mu_{0}^{2})}{2\sigma^{2}}\},\\
&= (\frac{1}{\sigma^{2}})^{\alpha + \frac{n}{2} + 1} \exp\{-\frac{2\beta + (\lambda + n)\mu^{2} - \mu(2\lambda\mu_{0} + 2\sum_{i=1}^{n} x_i) + \lambda\mu_{0}^{2} + \sum_{i=1}^{n} x_i^{2}}{2\sigma^{2}}\}.
\end{align*}
\end{enumerate}
Ignoring the terms not having $\mu$ or $\sigma^{2}$, 
\begin{align*}
f(\mu, \sigma^{2}| X)&\propto (\frac{1}{\sigma^{2}})^{\alpha + \frac{n}{2} + 1} \exp\{-\frac{2\beta + (\lambda + n)(\mu^{2} - 2\mu(\frac{\lambda\mu_{0} + 2\sum_{i=1}^{n} x_i }{\lambda + n}))}{2\sigma^{2}}\}.
\end{align*}
On completing square form,
\begin{align*}
f(\mu, \sigma^{2}| X)&\propto (\frac{1}{\sigma^{2}})^{\alpha + \frac{n}{2} + 1} \exp\{-\frac{2\beta + (\lambda + n)(\mu^{2} - 2\mu(\frac{\lambda\mu_{0} + 2\sum_{i=1}^{n} x_i }{\lambda + n}) + (\frac{\lambda\mu_{0} + 2\sum_{i=1}^{n} x_i }{\lambda + n})^{2} )}{2\sigma^{2}}\},\\
&(\frac{1}{\sigma^{2}})^{\alpha + \frac{n}{2} + 1} \exp\{-\frac{2\beta + (\lambda + n)(\mu - (\frac{\lambda\mu_{0} + 2\sum_{i=1}^{n} x_i }{\lambda + n}))^{2}}{2\sigma^{2}}\}.
\end{align*}
Normalizing the above term so that it is equal to one when integrated over sample space, we get
\begin{align*}
f(\mu, \sigma^{2}| X)&= \frac{\sqrt{\gamma}}{\sigma\sqrt{2\pi}} \frac{\beta^{\alpha}}{\Gamma(\alpha + \frac{n}{2})} (\frac{1}{\sigma^{2}})^{\alpha + \frac{n}{2} + 1}  \exp\{-\frac{2\beta + (\lambda + n)(\mu - (\frac{\lambda\mu_{0} + 2\sum_{i=1}^{n} x_i }{\lambda + n}))^{2}}{2\sigma^{2}}\},\\
&= \frac{\sqrt{\gamma}}{\sigma\sqrt{2\pi}} \frac{\beta^{\alpha}}{\Gamma(\alpha')} (\frac{1}{\sigma^{2}})^{\alpha' + 1}  \exp\{-\frac{2\beta + \lambda'(\mu - \mu')^{2}}{2\sigma^{2}}\}.
\end{align*}
which is posterior distribution that is in the same family as the prior distribution i.e., normal-inverse gamma distribution, where $\alpha' = \alpha + \frac{n}{2}$, $\lambda' = \lambda + n$, $\mu' = \frac{\lambda\mu_{0}\; +\; 2\sum_{i=1}^{n} x_i }{\lambda + n}$. Therefore a normal-inverse gamma distributions is conjugate prior for the class of normal distributions.
\hypertarget{solution6}{\subsection*{Solution 6}}
$X$ is a Bernoulli random variable with distribution $P(X=1)= 1-P(X=0)= p$.
\begin{equation*}
f(x_1,x_2,....,x_n|\;p)= \prod_{i=1}^{n}p^{x_i}(1-p)^{x_{i}}
\end{equation*}
and,
\begin{align*}
L(\theta|\bar{x})&= p^{\sum_{i=1}^{n}x_{i}} (1-p)^{n-\sum_{i=1}^{n}x_{i}},\\
&=p^{\alpha}(1-p)^{n-\alpha}\hspace{1cm}~(\mbox{where}~\alpha = \sum_{i=1}^{n}x_{i})
\end{align*}
Let
\begin{equation*}
y= \ln L(\theta|\bar{x}) = \alpha\ln p + (n-\alpha)\ln(1-p)
\end{equation*}
Differentiating $y$ w.r.t $p$ and equating it to zero, we get,
\begin{equation*}
\frac{\alpha}{p} + \frac{n-\alpha}{1-p}(-1)=0.
\end{equation*}
This gives $\hat{p} = \frac{\alpha}{n}$. Also,
\begin{equation*}
\frac{\partial^{2}y}{\partial p^{2}} = -\frac{\alpha}{p^{2}} - \frac{n-\alpha}{(1-p)^{2}}<0.
\end{equation*}
Hence MLE for $p$ is $\hat{p}=\frac{\alpha}{n}=\frac{1}{n}\sum_{i=1}^{n}x_{i}$.

\hypertarget{solution7}{\subsection*{Solution 7}}
Let $X_1,X_2....X_n$ be an iid random sample drawn from a Poisson distribution with parameter $\lambda,~X_{i}\sim\;poison(\lambda)$. Let $\bar X=\frac{1}{n}\sum_{i=1}^{n}x_{i}$, and $S^{2}=\frac{1}{n-1}\sum_{i=1}^{n}x_{i}^{2}$.
\begin{equation*}
\mathbb{E}[\bar{X}] = \frac{1}{n}(n\lambda) = \lambda.
\end{equation*}
\begin{align*}
\mathbb{E}[S^{2}] &= \frac{1}{n-1}\sum_{i=1}^{n} \;\mathbb{E}[x_{i}^{2}],\\
&=\frac{n(\lambda+\lambda^{2})}{n-1} = \tau(\lambda).
\end{align*}
We have,
\begin{align*}
Var(\bar{X})&= \frac{1}{n^{2}}
\sum_{i=1}^{n} Var(x_{i}),\\
&= \frac{n\lambda}{n^{2}} = \frac{\lambda}{n}.
\end{align*}
\begin{equation*}
Var(S^{2}) = \frac{1}{(n-1)^{2}} \sum_{i=1}^{n} Var(x_{i}^{2}),
\end{equation*}
\begin{equation*}
Var(x_{i}^{2}) = \mathbb{E}[x_{i}^{4}] - (\mathbb{E}[x_{i}^{2}])^{2},
\end{equation*}
\begin{equation*}
\mathbb{E}[x^{4}] = \sum_{k=0}^{\infty} k^{4} \frac{e^{-\lambda}\lambda^{k}}{k\,!}=e^{-\lambda}\sum_{k=1}^{\infty} \frac{k^{4}\lambda^{k}}{(k-1)\,!}.
\end{equation*}
For m=k-1,
\begin{align*}
E[x^{4}]&=e^{-\lambda}\sum_{m=0}^{\infty} \frac{(m+1)^{3}\lambda^{m+1}}{m\,!},\\
&=\lambda\sum_{m=0}^{\infty} e^{-\lambda}\frac{(m+1)^{3}\lambda^{m}}{m\,!},\\
&= \lambda\sum_{m=0}^{\infty} e^{-\lambda}\frac{(m^{3}+3m^{2}+3m+1)\lambda^{m}}{m\,!},\\
&=\lambda[E[x^{3}]+3E[x^{2}]+3E[x]+1],\\
&=\lambda[\lambda(E[x^{2}]+2E[X]+1)+3E[x^{2}]+3E[x]+1],\\
&=\lambda[\lambda(\lambda^{2}+\lambda+2\lambda+1)+3(\lambda^{2}+\lambda)+3\lambda+1].
\end{align*} 
On simplification, we get,
\begin{align*}
\mathbb{E}[x^{4}]=\lambda^{4}+6\lambda^{3}+7\lambda^{2}+\lambda,
\end{align*}
and
\begin{align*}
Var(x_i^{2})&=\mathbb{E}[x_{i}^{4}]-(\mathbb{E}[x_i^{2}])^{2},\\
&=\lambda^{4}+6\lambda^{3}+7\lambda^{2}+\lambda-(\lambda^{2}+\lambda)^{2},\\
&=4\lambda^{3}+6\lambda^{2}+\lambda.
\end{align*}
\begin{equation*}
Var(S^{2}) = \frac{n}{(n-1)^{2}} (4\lambda^{3} + 6\lambda^{2} + \lambda).
\end{equation*}
$Var(S^{2})$ is higher than var($\bar X$). Therefore $\bar X$ has lower mean square error.
\hypertarget{solution8}{\subsection*{Solution 8}}
Let $X_1,X_2,....X_n\;\epsilon\;R$ be an iid random sample, $X_i\sim N(\mu,\sigma^{2})$.
\subsubsection*{(a).}
Let $\bar X$ be an unbiased estimator of $\mu$. The pdf of normal distribution is,
\begin{equation*}
f_{X}(x;\mu) = \frac{1}{\sqrt{2\pi\sigma^{2}}} e^{-\frac{(x-\mu)^{2}}{2\sigma^{2}}}.
\end{equation*}
Cramer-Rao lower bound for unbiased estimator is $Var(\bar X)\geq \frac{1}{nFI}$. First we need to find Fisher Information (FI),
\begin{equation*}
FI = -\mathbb{E}\left[\frac{\partial^{2}}{\partial\mu^{2}} \ln f_{X}(x;\mu)\right].
\end{equation*}
Now,
\begin{equation*}
\ln f_{X}(x;\mu) = \ln\left(\frac{1}{\sqrt{2\pi\sigma^{2}}} e^{-\frac{(x-\mu)^{2}}{2\sigma^{2}}}\right)=-\ln\sigma\sqrt{2\pi} - \frac{(x-\mu)^{2}}{2\sigma^{2}}.
\end{equation*}
Differentiating twice, we get, $\frac{\partial^{2} \ln f(x;\mu)}{\partial \mu^{2}} = -\frac{1}{\sigma^{2}}$, hence, $FI =-\mathbb{E}[-\frac{1}{\sigma^{2}}] = \frac{1}{\sigma^{2}}$. Therefore, 
\begin{equation*}
Var(\bar X) \geq \frac{\sigma^{2}}{n}.
\end{equation*}
\subsubsection*{(b).}
$X_1,X_2....X_n\; \in \;R^{d}$, where $X_i \sim \mathcal{N}({\bf \mu},{\bf \Sigma}),~{\bf \Sigma}\in R^{d\times d}$ is the covariance matrix.
\begin{equation*}
P_{\bar X}(\bar x) = \frac{1}{(2\pi)^{\frac{n}{2}} {|\Sigma|^{\frac{1}{2}}}} e^{-\frac{1}{2} (\bar x - \bar \mu)^{T} \Sigma^{-1} (\bar x - \bar \mu)}
\end{equation*}
where $\bar{\mu} = \mathbb{E}[\bar X] $, $ \Sigma = \mathbb{E}[(\bar X - \bar \mu)^{T} (\bar X - \bar \mu)]$. A covariance matrix is always non-negative definite (i.e., $\bar x^{T}\Sigma \bar x \geq 0 )$. We assume that $\Sigma$ is positive definite (i.e., $\bar x^{T}\Sigma \bar x > 0$, except for $x=0$). The positive definite property of $\Sigma$ implies that $|\Sigma|>0$ and that $\Sigma^{-1}$ exists.
\begin{align*}
\ln P_{\bar X}(\bar x) &= -\frac{n}{2}\ln2\pi - \frac{1}{2}\ln | \Sigma | - \frac{1}{2}(\bar x- \bar{\mu})^{T}\Sigma^{-1}(\bar x- \bar \mu),\\
&= - \frac{n}{2}\ln2\pi - \frac{1}{2}\ln | \Sigma | - \frac{1}{2} (\bar x^{T}\Sigma^{-1} \bar x - \bar x^{T} \Sigma^{-1} \bar \mu - \bar \mu^{T} \Sigma^{-1} \bar \mu + \bar \mu^{T} \Sigma^{-1}\bar \mu),\\
&= - \frac{n}{2}\ln2\pi - \frac{1}{2}\ln | \Sigma | - \frac{1}{2} (\bar x^{T}\Sigma^{-1} \bar x - 2\bar x^{T} \Sigma^{-1} \bar \mu + \bar \mu^{T} \Sigma^{-1}\bar \mu).
\end{align*}
Differentiating $\ln P_{\bar X}(\bar x)$ with respect to $\mu$, we get,
\begin{equation*}
\frac{\partial}{\partial \mu}\ln P_{\bar X}(\bar x) = - \frac{1}{2} (-2 \bar \Sigma^{-1}\bar x + 2\Sigma^{-1}\bar\mu)=\bar \Sigma^{-1}\bar x - \Sigma^{-1}\bar\mu,
\end{equation*}
and,
\begin{equation*}
\frac{\partial^{2}}{\partial \mu^{2}} \ln P_{\bar X}(\bar x) = -\Sigma^{-1}.
\end{equation*}
Hence,
\begin{equation*}
-\mathbb{E}\left[\frac{\partial^{2}}{\partial \mu^{2}} \ln P_{\bar X}(\bar x)\right] = \Sigma^{-1}.
\end{equation*}
Therefore, 
\begin{equation*}
Var(\bar X) \geq (n\Sigma^{-1})^{-1}=\frac{1}{n}\Sigma
\end{equation*}
\hypertarget{solution9}{\subsection*{Solution 9}}
Consider the noisy signal model
\begin{equation}
Y = X + N~\mbox{where}~N\sim \mathcal{CN}(0,\sigma^{2}),~X = A\exp(j\phi)
\end{equation}
where, $A$ is the amplitude, and $\phi$ is the random phase uniformly distributed over  $[0,2\pi]$. Consider the real system model i.e.,
\begin{equation*}
Y_{1} = A\cos\phi + n_{1},~Y_{2} = A\sin\phi + n_{2},~~~(n_{1},n_{2} \sim N(0,\sigma^{2})).
\end{equation*}
The density of $Y$ can be written as,
\begin{equation*}
P_{\Theta}(Y) = \frac{1}{2\pi \sigma^{2}} \exp(-\frac{q(y,\phi)}{2\sigma^{2}})
\end{equation*} 
where $q(y,\phi) = [(y_{1} - A\cos\phi)^{2} + (y_{2} - A\sin\phi)^{2}]$. The marginal density of $Y$ is,
\begin{align*}
P(Y|A) &= \frac{1}{2\pi} \int_{0}^{2\pi} P_{\Theta}(Y) \;d\phi,\\
&=\frac{1}{4\pi^{2} \sigma^{2}} \int_{0}^{2\pi} \exp\left[-\frac{q(y,\phi)}{2\sigma^{2}}\right] \;d\phi,\\
&=\frac{1}{4\pi^{2} \sigma^{2}} \int_{0}^{2\pi} \exp\{-\frac{(y_{1} - A\cos\phi)^{2} + (y_{2} - A\sin\phi)^{2}}{2\sigma^{2}}\} \;d\phi,\\
&= \frac{1}{2\pi} \frac{\exp\{-\frac{1}{2\sigma^{2}}(y_{1}^{2} + y_{2}^{2} + A^{2})\}}{2\pi \sigma^{2}} \int_{0}^{2\pi} \exp\{\frac{A}{\sigma^{2}}(y_{1}\cos\phi + y_{2}\sin \phi)\} \;d\phi,\\
&= \frac{1}{2\pi} \frac{\exp\{-\frac{1}{2\sigma^{2}}(y_{1}^{2} + y_{2}^{2} + A^{2})\}}{2\pi \sigma^{2}}  \int_{0}^{2\pi}\exp \{\frac{Ar}{\sigma^{2}}\cos(\phi - \theta)\} \;d\phi,\\
&= \frac{1}{2\pi} \frac{\exp\{-\frac{1}{2\sigma^{2}}(y_{1}^{2} + y_{2}^{2} + A^{2})\}}{2\pi \sigma^{2}} I_0(\frac{Ar}{\sigma^{2}}).
\end{align*}
where $r = \sqrt{y_{1}^{2} + y_{2}^{2}}$, $\theta  = \tan^{-1}(\frac{y_{2}}{y_{1}})$, $y_{1} = r\cos \phi$, $y_{2} = r\sin\phi$, and Bessel function of zero'th order,
\begin{equation*}
I_{0}(x) = \frac{1}{2\pi} \int_{0}^{2\pi} \exp[x\cos \tau] \;d\tau.
\end{equation*}
Using approximation $I_{0}(x) \simeq \frac{e^{x}}{\sqrt{2\pi x}}$, we get
\begin{equation*}
P(Y|A) \approx \frac{1}{2\pi} \frac{\exp\{-\frac{1}{2\sigma^{2}}(y_{1}^{2} + y_{2}^{2} + A^{2})\}}{2\pi \sigma^{2}}  \frac{e^{\frac{Ar}{\sigma^{2}}}}{\sqrt{2\pi(\frac{Ar}{\sigma^{2}})}}
\end{equation*}
Differentiating $P(Y|\theta_{1})$ with respect to $A$ and equating it to zero, we get the maximum likelihood estimate of $A$ as $\frac{r+\sqrt{r^{2} - 2\sigma^{2}}}{2}$.

\end{document}
