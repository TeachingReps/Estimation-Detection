\documentclass[]{article}

%opening
\title{Homework-3 Solutions}
\author{}
\usepackage{graphicx}
\begin{document}

\maketitle 
\begin{flushleft}
$1)\;$ Given  \\
\end{flushleft}	
$
\hspace{3cm}p_0(y) =
\left\{
\begin{array}{lr}
1& 0 \le y \le 1 \\
0& $otherwise$
\end{array}
\right.
$    \\

and \\

$
\hspace{2.5cm}p_1(y)=
\left\{
\begin{array}{lr}
2y& 0 \le y \le 1 \\
0& $otherwise$
\end{array}
\right.
$\\

$
 Y_1,Y_2,.....,Y_n$ are iid \\
 
$\hspace{1cm} P_F(\delta)\leq\exp(-s\tau+\mu_{T,0}(s)) $ \\

$ \hspace{1cm}P_M(\delta)\leq\exp(\tau)\exp(-s\tau+\mu_{T,0}(s)) $  \\

where 
$\hspace{0.5cm}\mu_{T,0}(s)=\log E_0[e^{(s\log L(y)}] $\\

$\hspace{2.75cm}=\log E_0[L(y)^{s}] $ \\ 

The Likelihood Ratio is \\

$ \hspace{1cm}L(y)=\prod_{i=1}^{n}\frac{p_1(y_{i})}{p_0(y_{i})} $
$ =\prod_{i=1}^{n} 2y_{i} $
$ =2^{n}\prod_{i=1}^{n} y_{i} $ \\

$\hspace{1cm} E_0[L(y)^{s}] = 2^{ns} (E_{0}[y_{1}^{s}])^{n} $
$ = 2^{ns}(\int_{0}^{1}y^{ns}ds)^{n} $ 
$ = \frac{2^{ns}}{n(s + 1)} $ \\

$\hspace{1cm} \mu_{T,0}(s) = ns\log2 - n\log(s + 1) $ \\

Case1: $\;\tau<n\log2 $\\

$ \hspace{1cm}P_F(\delta)\leq 1 $ \\

$ \hspace{1cm}P_M(\delta)\leq \exp(\tau) $ \\

 Case2: $\;\tau>n\log2  $\\

$ \hspace{1cm}P_F(\delta)\leq \exp(-\tau) $\\
 
$ \hspace{1cm}P_M(\delta)\leq 1 $ \\
  
\begin{flushleft}
$2)$ Given
\end{flushleft}

$Y_1,Y_2,.....$ are iid Bernouli observations with distribution
\begin{center}
$H_0\;: P(Y_k=1)=1-P(Y_k=0)= \frac{1}{3}$\\
$H_1\;: P(Y_k=1)=1-P(Y_k=0)= \frac{2}{3}$
\end{center}


$a)\;$From Wald's approximation,\\

$\hspace{1cm}$A$\cong\frac{\gamma}{1-\alpha}$\\

$\hspace{1cm}$B$\cong\frac{1-\gamma}{\alpha}$\\

$\hspace{0.5cm}\Rightarrow A\leq\frac{0.01}{0.99} \;\;\Rightarrow\; A\leq\frac{1}{99}$\\

$\hspace{1cm}B\geq\frac{0.99}{0.01} \;\;\Rightarrow\; B\geq99$\\

The likelihood ratio is given by $$\; L(Y_k)= \log\frac{P(Y_k\mid\;H_1)}{P(Y_k\mid\;H_0)} =
\left\{
\begin{array}{lr}
\log2&;Y_k=1 \\
-\log2&;Y_k=0
\end{array}
\right.$$

The SPRT(Sequential Probability Ratio Test) compares the cummulative 

likelihood ratio test to an upper and lower threshold. It declares one of the 

hypothesis when the hypothesis crosses the corresponding threshold.\\


$b)\;$ Consider hypothesis $\;H_1$\\


By Chernoff Bound 
$\gamma \simeq\;\exp(-2n(P-\frac{1}{2})^{2})
$ where $P=P(Y_k=1\mid\;H_1)=\frac{2}{3}$

Hence $\exp(-10) = \;\exp(-\frac{n}{18})$ giving $n = 180 \;$samples\\

$ c)\hspace{0.5cm}P^{*}=max(P_{F},P_{M})$=0.01


$\hspace{0.9cm}\alpha=P_{F}$

$\hspace{0.9cm}\gamma=P_{M}$

$\hspace{0.5cm}\Rightarrow max(\alpha,\gamma)=0.01$\\



If we choose $A\leq\frac{1}{99}$ and $B\geq99,$ we can always have $P^{*}=0.01$\\

$\hspace{1cm}$E$[N\mid H_0]= (\frac{1}{\mu_0})[(1-\alpha)\log(\frac{\gamma}{1-\alpha})+\alpha\;\log(\frac{1-\gamma}{\alpha})]$\\

$\hspace{1cm}$E$[N\mid H_1]= (\frac{1}{\mu_1})[\gamma\;\log(\frac{\gamma}{1-\alpha})+(1-\gamma)\log(\frac{1-\gamma}{\alpha})]$\\

$\hspace{1cm}$$\mu_{j}=E[\log(\frac{P_{1}(Y_{1})}{P_{0}(Y_{1})})\mid\;H_{j}]$\\

$\hspace{1cm}\mu_{0}=\frac{1}{3}\log(\frac{\frac{2}{3}}{\frac{1}{3}}) +\frac{2}{3}\log(\frac{\frac{1}{3}}{\frac{2}{3}})$\\

$\hspace{1.5cm}=\frac{1}{3}\log2 - \frac{2}{3} \log2$\\

$\hspace{1.5cm}= - \frac{1}{3} \log2$\\

$\hspace{1cm}\mu_{1}=\frac{2}{3}\log(\frac{\frac{2}{3}}{\frac{1}{3}}) +\frac{1}{3}\log(\frac{\frac{1}{3}}{\frac{2}{3}})$\\

$\hspace{1.5cm}=\frac{2}{3}\log2 - \frac{1}{3} \log2$\\

$\hspace{1.5cm}= \frac{1}{3} \log2$\\

Choose$ \;A=\frac{1}{99},B=99 \Rightarrow \alpha=\gamma=0.01$\\

Then $\;$$E[N\mid H_0]= \frac{-3}{\log2}[0.99\log(\frac{0.01}{0.99})+0.01\log(\frac{0.99}{0.01})]$\\

$\hspace{2.7cm}$$=\frac{-3}{\log2}[0.99\log(\frac{1}{99})+0.01\log(99)]$\\

$\hspace{2.7cm}$$=\frac{-3}{\log2}[-0.99\log(99)+0.01\log(99)]$\\

$\hspace{2.7cm}$$=\frac{3\times0.98}{\log2}\;\log(99)$\\

$\hspace{2.7cm}$$=19.49$\\

$\hspace{1cm}$$E[N\mid H_1]= \frac{3}{log2}[0.01\log(\frac{0.01}{0.99})+0.99
\log(\frac{0.99}{0.01})]$\\

$\hspace{2.7cm}$$=\frac{3}{\log2}[-0.01\log(99)+0.99\log(99)]$\\

$\hspace{2.7cm}$$=\frac{2.94\;\log(99)}{\log2}$\\

$\hspace{2.7cm}$$=19.49$\\

The calculated values of $E[N\mid H_0]$ and $E[N\mid H_1]$
will vary depending on 

the exact $\alpha$,$\gamma$ we choose and in effect the A and B choosen.

\begin{flushleft}
3) Let $\bar{\phi}$ = $\{\phi_{j};j=0,1,2,...\}$ be a stopping rule and\\

$\hspace{1cm}\bar{\delta}$ = $\{\delta_{j};j=0,1,2,...\}$ be a terminal decision rule
\end{flushleft}

A sequential decision rule is pair of sequences$(\bar{\phi},\bar{\delta})$ the rule $(\bar{\phi},\bar{\delta})$ makes the 

decision $\delta_{N}(y_{1},y_{2},...y_{N})$ where N is the stopping time defined by

$N=\min\{n\mid \phi_{n}(y_{1},y_{2},...y_{n})=1\}$\\

The sequential decision rule of n-samples is defined by\\

$
\hspace{2.5cm}\phi_{j}(y_{1},...,y_{j})=
\left\{
\begin{array}{lr}
0& $\hspace{1.5cm}$   ; j\neq n \\
1&  $\hspace{1.5cm}$     ;  j=n
\end{array}
\right.
$\\

$
\hspace{2.5cm}\delta_{j}(y_{1},...,y_{j})=
\left\{
\begin{array}{lr}
\delta(y_{1},...,y_{n})& ; j= n \\
1& ; j\neq n
\end{array}
\right.
$\\

The priors $ \pi_{1}$ and $ \pi_{0}$ are assigned to hypothesis $ H_{1} $ and $H_{0},$ respectively

the Bayes risk is given by\\

$\hspace{2cm}$
$r(\bar{\phi},\bar{\delta})=(1-\pi_{1})R_{0}(\bar{\phi},\bar{\delta})+\pi_{1}R_{1}(\bar{\phi},\bar{\delta})$\\

Consider the function\\

$\hspace{2cm}$
$V^{*}(\pi_{1})$=min $r(\bar{\phi},\bar{\delta})$\\

The relationship between Bayes risk and uniform costs has been divided

into two decisions rules.
The one that takes no samples and decides $H_{1}$

$($i.e., $\phi_{0}=\delta_{0}=1)$ and the one that takes no samples and decides $H_{0}$

$($i.e., $\phi_{0}=1-\delta_{0}=1)$ \\

$ \hspace{2cm}$ $r(\bar{\phi},\bar{\delta})\mid_{\phi_{0}=\delta_{0}=1}=(1-\pi_{1})$

and

$ \hspace{2cm} $
$r(\bar{\phi},\bar{\delta})\mid_{\phi_{0}=1-\delta_{0}=1}=\pi_{1} $ \\

\begin{figure}[ht!]
\centering
\includegraphics[width=90mm]{1.png}
\caption{Relationships yielding the Bayes sequential rule for uniform
costs of errors and cost С per sample. \label{overflow}}
\end{figure}

By inspection of the figure, we see that Bayes rule for a fixed prior $\pi_{1}$ is \\

$
\hspace{2.5cm}\phi_{0}=
\left\{
\begin{array}{lr}
1-\delta_{0} = 1& \pi_{1}\leq\pi_{L} \\
\delta_{0} = 1& \pi_{L}\geq\pi_{U}
\end{array}
\right.
$\\ 

and it is the decision rule with the minimum Bayes risk among all $(\bar \phi,\bar \delta)$ with

$\phi_{0}$ if $\pi_{L}<\pi_{1}<\pi_{U} $. So if $\pi_{1}\leq\pi_{L}$ we take no samples and choose $H_{0}$, if

$\pi_{1}\geq \pi_{U}$ we take no samples and choose $H_{1}$. Otherwise we take atleast one 

sample.

\begin{flushleft}
4) Let $X_1,X_2,....X_n$ be a random sample
\end{flushleft}

$\hspace{1cm}f(x\mid\theta)=\theta\;x^{-2} \; ;\;0<\theta\leq\;x<\infty$\\

The maximum likelihood estimator of $\theta$ is $L_\theta(x) = \prod_{i=1}^{n} f(x_i\mid\theta)$\\

a)$\;\;\;f(\bar{x}\mid\theta)= \prod_{i=1}^{n} f(x_i\mid\theta)$\\

$\hspace{1.85cm}=\prod_{i=1}^{n} \theta\;x_{i}^{-2} I_{[\theta,\infty]}(x_{i})$\\

$\hspace{1.85cm}=(\prod_{i=1}^{n} \;x_{i}^{-2})\theta^{n}I_{[\theta,\infty]}(x_{(1)}) \hspace{1cm} ( x_{(1)}$ is minimum of X)\\

Hence by factorzation theorem $X_{(1)}$ is a sufficient statistic for $\theta$\\

b) $\;L(\theta\mid\bar{x}) = \theta^{n}(\prod_{i=1}^{n} \;x_{i}^{-2})I_{[\theta,\infty]}(x_{(1)})$\\

$\theta^{n}$ is increasing in $\theta$. The second term of above equation doesn't depend on

$\theta$. So to maximize the MLE $L(\theta\mid\bar x)$, we want to make $\theta$ as large as possible.

But because of the indicator function\\

$\hspace{2cm}L(\theta\mid\bar x)$ = 0 ; $\theta>x_{(1)}$\\

Thus, $\hspace{1cm}\hat{\theta} = x_{(1)}$\\

c) $\hspace{1cm}$E[X] = $\int_\theta^{\infty}\theta x^{-1}\,dx = \infty $ \\

Thus the method of moments estimator of $ \theta $  doesn't exit.\\

\begin{flushleft}
5) a) If the posterior distributions are in the same family as the prior  
\end{flushleft} 

probability distribution, the prior and posterior are then called conjugate 

distributions, and the prior is called a conjugate prior for the likelihood func-

tion.\\

By Bayes' rule $P(\mu \mid x) = \frac{P(x\mid \mu)P(\mu)}{P(x)}$\\

Since P(x) is independent of $\mu$ the above equation can be written as\\

$\hspace{3cm}P(\mu \mid x) \propto P(x\mid \mu)P(\mu)$\\

i.e., Posterior distribution $\propto$ Likelihood $\times$ Prior distribution\\

$X_1,X_2,....,X_n$ is a random sample with $X_i \sim N(\mu,\sigma_{x}^{2})$

Likelihood function, 

$\hspace{3cm}P(X \mid \mu) = \prod_{i=1}^{n} \frac{1}{\sqrt{2\pi \sigma^{2}}} \;e^{-\frac{(x_{i}-\mu)^{2}}{2\sigma_{x}^{2}}}$

Prior function,

$\hspace{3.5cm} P(\mu) = \frac{1}{\sqrt{2\pi \sigma^{2}}} e^{-\frac{(\mu-\mu_{0})^{2}}{2\sigma_{\mu}^{2}}}$\\

Likelihood $\times$ Prior = $\prod_{i=1}^{n} \frac{1}{\sqrt{2\pi \sigma^{2}}} \;e^{-\frac{(x_{i}-\mu)^{2}}{2\sigma_{x}^{2}}} \times \frac{1}{\sqrt{2\pi \sigma^{2}}} e^{\frac{(\mu-\mu_{0})^{2}}{2\sigma_{\mu}^{2}}} $\\

$\hspace{3cm} = \frac{1}{(\sqrt{2\pi \sigma^{2}})^{n}} \;e^{-\frac{\sum_{i=1}^{n}(x_{i}-\mu)^{2}}{2\sigma_{x}^{2}}} \times \frac{1}{\sqrt{2\pi \sigma^{2}}} e^{-\frac{(\mu-\mu_{0})^{2}}{2\sigma_{\mu}^{2}}}$\\

$\hspace{3cm} \propto \;e^{-\frac{\sum_{i=1}^{n}(x_{i}-\mu)^{2}}{2\sigma_{x}^{2}}} \times e^{-\frac{(\mu-\mu_{0})^{2}}{2\sigma_{\mu}^{2}}}$ $\hspace{0.5cm}$(constants are ignored)\\

$\hspace{3cm} \propto \exp[{\frac{-\sum_{i=1}^{n}x_{i}^{2}+2\mu n \bar X-n\mu^{2}}{2\sigma_{x}^{2}}}-\frac{\mu^{2} - 2\mu \mu_{0}+\mu_{0^{2}}}{2\sigma_{\mu}^{2}}]$\\

We are interested only in distribution of $\mu$, neglect the terms not having $\mu$\\

$\hspace{3cm} \propto \exp[\frac{2\mu n\bar X -n\mu^{2}}{2\sigma_{x}^{2}} - \frac{\mu^{2} - 2\mu \mu_{0}}{2\sigma_{\mu}^{2}}]$\\

$\hspace{3cm} \propto \exp[-\frac{\mu^{2}}{2}(\frac{1}{\sigma_{\mu}^{2}}+\frac{n}{\sigma_{x}^{2}}) - \mu(\frac{\mu_{0}}{\sigma_{\mu}^{2}}+\frac{n\bar X}{\sigma_{x}^{2}}]) \hspace{2cm} (1)$\\

Let  $\hspace{0.3cm}\sigma_{\mu}^{'^{2}} = [\frac{1}{\sigma_{\mu}^{2}}+\frac{n}{\sigma_{x}^{2}}]^{-1}$\\

$\hspace{1cm} \mu^{'} = \sigma_{\mu}^{'^{2}}[\frac{\mu_{0}}{\sigma_{\mu}^{2}}+\frac{n\bar X}{\sigma_{x}^{2}}]$\\

Substitute $\sigma_{\mu}^{'^{2}}, \mu^{'}$ in (1)\\

$\hspace{3cm}\propto \exp[-\frac{\mu^{2}}{2\sigma_{\mu}^{'^{2}}}+\frac{\mu \mu^{'}}{\sigma_{\mu}^{'^{2}}}]$\\

Multiply the above term by a constant\\

$\hspace{3cm} = \exp[-\frac{\mu^{2}}{2\sigma_{\mu}^{'^{2}}}+\frac{\mu \mu^{'}}{\sigma_{\mu}^{'^{2}}}-\frac{\mu^{'^{2}}}{2\sigma_{\mu}^{'^{2}}}] \hspace{3.5cm} (2)$\\

$\hspace{3cm} = \exp[-\frac{(\mu-\mu^{'})^{2}}{2\sigma_{\mu}^{2}}]$

Normalizing (2) so that it is equal to one when integrated over sample space, 

we get

$\hspace{3cm} = \frac{1}{\sqrt{2\pi \sigma_{\mu}^{'^{2}}}} e^{-\frac{(\mu-\mu^{'})^{2}}{2\sigma_{\mu}^{2}}}$\\

which is posterior distribution that is in the same family as prior distribution.

Therefore the class of normal distributions is a conjugate prior family for 

the class of normal distributions with known variance\\ 

b) The likelihood function a normally distributed random sample is 

$\hspace{3cm}f(X \mid \mu) = \prod_{i=1}^{n} \frac{1}{\sqrt{2\pi \sigma^{2}}} \;e^{-\frac{(x_{i}-\mu)^{2}}{2\sigma^{2}}}$\\

The normal inverse-gamma distribution has the form\\

$\hspace{3cm} f(\mu, \sigma^{2}\mid \mu_{0}, \lambda,\alpha,\beta) =  \frac{\sqrt{\gamma}}{\sigma\sqrt{2\pi}} \frac{\beta^{\alpha}}{\Gamma(\alpha)} (\frac{1}{\sigma^{2}})^{\alpha + 1} \exp(-\frac{2\beta + \lambda(\mu - \mu_{0})^{2}}{2\sigma^{2}}) $\\

Posterior $\propto$ Likelihood $\times$ Prior 

Likelihood $\times$ Prior = $\frac{1}{(\sqrt{2\pi \sigma^{2}})^{n}} \;\exp\{-\frac{\sum_{i=1}^{n}(x_{i}-\mu)^{2}}{2\sigma^{2}}\} \times \frac{\sqrt{\gamma}}{\sigma\sqrt{2\pi}} \frac{\beta^{\alpha}}{\Gamma(\alpha)} (\frac{1}{\sigma^{2}})^{\alpha + 1} \exp\{-\frac{2\beta + \lambda(\mu - \mu_{0})^{2}}{2\sigma^{2}}\}$\\

$\hspace{2.9cm} \propto (\frac{1}{\sigma^{2}})^{\alpha + \frac{n}{2} + 1} \exp\{-\frac{2\beta + \sum_{i=1}^{n}(x_{i}-\mu)^{2} + \lambda(\mu - \mu_{0})^{2}}{2\sigma^{2}}\}$\\

$\hspace{2.9cm} \propto (\frac{1}{\sigma^{2}})^{\alpha + \frac{n}{2} + 1} \exp\{-\frac{2\beta + \sum_{i=1}^{n}x_{i}^{2}- 2\mu\sum_{i=1}^{n}x_{i}+ n\mu^{2} + \lambda(\mu^{2} - 2\mu\mu_{0} + \mu_{0}^{2})}{2\sigma^{2}}\}$\\

$\hspace{2.9cm} \propto (\frac{1}{\sigma^{2}})^{\alpha + \frac{n}{2} + 1} \exp\{-\frac{2\beta + (\lambda + n)\mu^{2} - \mu(2\lambda\mu_{0} + 2\sum_{i=1}^{n} x_i) + \lambda\mu_{0}^{2} + \sum_{i=1}^{n} x_i^{2}}{2\sigma^{2}}\}$\\

Ignore the terms not having $\mu$ or $\sigma^{2}$

$\hspace{2.9cm} \propto (\frac{1}{\sigma^{2}})^{\alpha + \frac{n}{2} + 1} \exp\{-\frac{2\beta + (\lambda + n)(\mu^{2} - 2\mu(\frac{\lambda\mu_{0} + 2\sum_{i=1}^{n} x_i }{\lambda + n}))}{2\sigma^{2}}\}$\\

Multiply the above term by a constant

$\hspace{2.9cm} = (\frac{1}{\sigma^{2}})^{\alpha + \frac{n}{2} + 1} \exp\{-\frac{2\beta + (\lambda + n)(\mu^{2} - 2\mu(\frac{\lambda\mu_{0} + 2\sum_{i=1}^{n} x_i }{\lambda + n}) + (\frac{\lambda\mu_{0} + 2\sum_{i=1}^{n} x_i }{\lambda + n})^{2} )}{2\sigma^{2}}\}$

$\hspace{2.9cm} = (\frac{1}{\sigma^{2}})^{\alpha + \frac{n}{2} + 1} \exp\{-\frac{2\beta + (\lambda + n)(\mu - (\frac{\lambda\mu_{0} + 2\sum_{i=1}^{n} x_i }{\lambda + n}))^{2}}{2\sigma^{2}}\}$\\

Normalizing the above term so that it is equal to one when integrated over

sample space, we get\\

$\hspace{2.9cm} = \frac{\sqrt{\gamma}}{\sigma\sqrt{2\pi}} \frac{\beta^{\alpha}}{\Gamma(\alpha + \frac{n}{2})} (\frac{1}{\sigma^{2}})^{\alpha + \frac{n}{2} + 1}  \exp\{-\frac{2\beta + (\lambda + n)(\mu - (\frac{\lambda\mu_{0} + 2\sum_{i=1}^{n} x_i }{\lambda + n}))^{2}}{2\sigma^{2}}\}$\\

$\hspace{2.9cm} = \frac{\sqrt{\gamma}}{\sigma\sqrt{2\pi}} \frac{\beta^{\alpha}}{\Gamma(\alpha^{'})} (\frac{1}{\sigma^{2}})^{\alpha^{'} + 1}  \exp\{-\frac{2\beta + \lambda^{'}(\mu - \mu^{'})^{2}}{2\sigma^{2}}\}$\\

which is posterior distribution that is in the same family as the prior distri-

bution i.e., normal-inverse gamma distribution.\\

where $\hspace{1cm}\alpha^{'} = \alpha + \frac{n}{2}$\\

$\hspace{2cm} \lambda^{'} = \lambda + n$\\

$\hspace{1cm} \mu^{'} = \frac{\lambda\mu_{0}\; +\; 2\sum_{i=1}^{n} x_i }{\lambda + n}$\\

Therefore a normal-inverse gamma distributions is conjugate prior for the 

class of normal distributions.

\begin{flushleft}
6) X is a Bernouli random variable with distribution
\end{flushleft}

\begin{center}
$P(X=1)= 1-P(X=0)= p$
\end{center}

$\hspace{2cm}$f$(x_1,x_2,....,x_n\mid\;p)= \prod_{i=1}^{n}p^{x_i}(1-p)^{x_{i}}$\\

$\hspace{2cm} L(\theta\mid\bar{x})= p^{\sum_{i=1}^{n}x_{i}} (1-p)^{n-\sum_{i=1}^{n}x_{i}}$\\

$\hspace{3cm} = p^{\alpha}(1-p)^{n-\alpha}$ $\hspace{1cm}$(where $\alpha = \sum_{i=1}^{n}x_{i}$)\\

$\hspace{2cm}y= \ln L(\theta\mid\bar{x}) = \alpha\ln p + (n-\alpha)\ln(1-p)$\\

Differentiating $y$ w.r.t $p$ and equating it to zero\\

$\hspace{2cm} \frac{\partial y}{\partial p} = 0$\\

$\hspace{1.5cm}\Rightarrow \frac{\alpha}{p} + \frac{n-\alpha}{1-p}(-1)=0$\\

$\hspace{1.5cm} \Rightarrow \alpha = np$\\

$\hspace{2cm}  p = \frac{n}{\alpha}$\\

$\hspace{1.5cm} \frac{\partial^{2}y}{\partial p^{2}} = -\frac{\alpha}{p^{2}} - \frac{n-\alpha}{(1-p)^{2}}<0$ \\

Hence MLE for $p$ is $\frac{\alpha}{n}=\frac{\sum_{i=1}^{n}x_{i}}{n}$



\begin{flushleft}
7) Let $X_1,X_2....X_n$ be an iid random sample drawn from a Poisson 
\end{flushleft}

distribution with parameter $\lambda, X_{i}\sim\;poison(\lambda)$\\

$\hspace{1cm}\bar X=\frac{1}{n}\sum_{i=1}^{n} \;x_{i}$\\

$\hspace{1cm}S^{2}=\frac{1}{n-1}\sum_{i=1}^{n} \;x_{i}^{2}$\\

$\hspace{1cm}E[\bar{X}] = \frac{1}{n}(n\lambda) = \lambda$\\

$\hspace{1cm}E[S^{2}] = \frac{1}{n-1}\sum_{i=1}^{n} \;E[x_{i}^{2}]$\\

$\hspace{2cm}=\frac{n(\lambda+\lambda^{2})}{n-1} = \tau(\lambda)$\\

$\hspace{1cm}$var$(\bar{X})= \frac{1}{n^{2}}
\sum_{i=1}^{n} $var$(x_{i})$\\

$\hspace{2cm} = \frac{n\lambda}{n^{2}} = \frac{\lambda}{n}$\\

$\hspace{1cm}$ var$(S^{2}) = \frac{1}{(n-1)^{2}} \sum_{i=1}^{n} $var$(x_{i}^{2})$\\

$\hspace{1cm}$var$(x_{i}^{2})$ = E$[x_{i}^{4}]$ - (E$[x_{i}^{2}])^{2}$\\

$\hspace{1cm}$ E$[x^{4}] = \sum_{k=0}^{\infty} k^{4} \frac{e^{-\lambda}\lambda^{k}}{k\,!}$\\

$\hspace{2cm}$ =$ e^{-\lambda}\sum_{k=1}^{\infty} \frac{k^{4}\lambda^{k}}{(k-1)\,!}$\\

for m=k-1\\
 
$\hspace{1cm}$ E$[x^{4}]$ = $e^{-\lambda}\sum_{m=0}^{\infty} \frac{(m+1)^{3}\lambda^{m+1}}{m\,!}$ \\
 
 
$ \hspace{2cm} $= $\lambda\sum_{m=0}^{\infty} e^{-\lambda}\frac{(m+1)^{3}\lambda^{m}}{m\,!}$ \\
 
$ \hspace{2cm} $= $\lambda\sum_{m=0}^{\infty} e^{-\lambda}\frac{(m^{3}+3m^{2}+3m+1)\lambda^{m}}{m\,!}$ \\
 

$ \hspace{2cm} $= $\lambda[E[x^{3}]+3E[x^{2}]+3E[x]+1]$ \\

$ \hspace{2cm} $= $\lambda[\lambda(E[x^{2}]+2E[X]+1)+3E[x^{2}]+3E[x]+1]$ \\

$ \hspace{2cm} $=
$\lambda[\lambda(\lambda^{2}+\lambda+2\lambda+1)+3(\lambda^{2}+\lambda)+3\lambda+1]$ \\

By simplifying we will get\\

$\hspace{1cm}$ E$[x^{4}]$=$ \lambda^{4}+6\lambda^{3}+7\lambda^{2}+\lambda $\\

$\hspace{1cm}$ var$(x_i^{2})$= E$[x_{i}^{4}]$-(E$[x_i^{2}])^{2} $ \\
 
$\hspace{2cm}$=$\lambda^{4}+6\lambda^{3}+7\lambda^{2}+\lambda-(\lambda^{2}+\lambda)^{2} $\\
 
$\hspace{2cm}$=$4\lambda^{3}+6\lambda^{2}+\lambda$\\

$\hspace{0.8cm}\Rightarrow$ var$(S^{2}) = \frac{n}{(n-1)^{2}} (4\lambda^{3} + 6\lambda^{2} + \lambda) $\\

var($S^{2}$) is higher than var($\bar X$)

Therefore $\bar X$ has lower mean square error

\begin{flushleft}
8) Let $X_1,X_2,....X_n\;\epsilon\;R$ be an iid random sample
\end{flushleft}

$\hspace{2cm}X_i\sim N(\mu,\sigma^{2})$\\

a) $\bar X$ be an unbiased estimator of $\mu$\\

The pdf of normal distribution is\\

$\hspace{2cm} f_{X}(x;\mu) = \frac{1}{\sqrt{2\pi\sigma^{2}}} e^{-\frac{(x-\mu)^{2}}{2\sigma^{2}}}$\\

Cramer-Rao lower bound for unbiased estmator is\\

$\hspace{2cm}$var($\bar X$) $\geq \frac{1}{nFI}$\\

First we need to find Fisher Information(FI)\\

$\hspace{2cm}$ FI = $-E[\frac{\partial^{2}}{\partial\mu^{2}} \ln f_{X}(x;\mu)]$

$\hspace{1.5cm} \ln f_{X}(x;\mu) = \ln(\frac{1}{\sqrt{2\pi\sigma^{2}}} e^{-\frac{(x-\mu)^{2}}{2\sigma^{2}}}) $

$\hspace{3.25cm} = -\ln\sigma\sqrt{2\pi} - \frac{(x-\mu)^{2}}{2\sigma^{2}}$\\

$\hspace{1.75cm} \frac{\partial \ln f(x;\mu)}{\partial \mu} = \frac{(x-\mu)^{2}}{\sigma^{2}}$

$\hspace{1.65cm} \frac{\partial^{2} \ln f(x;\mu)}{\partial \mu^{2}} = -\frac{1}{\sigma^{2}}$\\

$\hspace{2cm}$ FI = -$E[-\frac{1}{\sigma^{2}}] = \frac{1}{\sigma^{2}}$

Therefore, 

$\hspace{2cm}$ var($\bar X) \geq \frac{\sigma^{2}}{n}$\\

b) $X_1,X_2....X_n\; \epsilon \;R^{d}$ where $X_i \sim N(\textbf{$\mu$},\textbf{$\Sigma$}),\; \textbf{$\Sigma$}\; \epsilon \;R^{d\times d} $ is the covariance matrix\\

$\hspace{2cm}$$ P_{\bar X}(\bar x) = \frac{1}{(2\pi)^{\frac{n}{2}} {\mid \Sigma \mid ^{\frac{1}{2}}}} e^{-\frac{1}{2} (\bar x - \bar \mu)^{T} \Sigma^{-1} (\bar x - \bar \mu)} $\\

where $\hspace{3cm}$$ \bar{\mu} = E[\bar X] $\\

$\hspace{2.85cm}$ $ \Sigma = E[(\bar X - \bar \mu)^{T} (\bar X - \bar \mu)]$\\

A covariance matrix is always non-negative definite(i.e., $\bar x^{T}\Sigma \bar x \geq 0 )$. We 

assume that $\Sigma$ is positive definite (i.e., $\bar x^{T}\Sigma \bar x > 0$, except for x=0$)$. The 

positive definite of $\Sigma$ implies that $\mid \Sigma \mid$$>$0 and that $\Sigma^{-1}$ exists.\\

$\hspace{1cm}$ $\ln P_{\bar X}(\bar x) = -\frac{n}{2}\ln2\pi - \frac{1}{2}\ln \mid \Sigma \mid - \frac{1}{2}(\bar x- \bar{\mu})^{T}\Sigma^{-1}(\bar x- \bar \mu)$\\

$\hspace{1cm}$ $ = - \frac{n}{2}\ln2\pi - \frac{1}{2}\ln \mid \Sigma \mid - \frac{1}{2} (\bar x^{T}\Sigma^{-1} \bar x - \bar x^{T} \Sigma^{-1} \bar \mu - \bar \mu^{T} \Sigma^{-1} \bar \mu + \bar \mu^{T} \Sigma^{-1}\bar \mu)$\\

$\hspace{1cm}$ $ = - \frac{n}{2}\ln2\pi - \frac{1}{2}\ln \mid \Sigma \mid - \frac{1}{2} (\bar x^{T}\Sigma^{-1} \bar x - 2\bar x^{T} \Sigma^{-1} \bar \mu + \bar \mu^{T} \Sigma^{-1}\bar \mu)$\\

Differentiating $\ln P_{\bar X}(\bar x)$ with respect to $\mu$\\

$\hspace{1cm}\frac{\partial}{\partial \mu}\ln P_{\bar X}(\bar x) = - 0 - 0 - \frac{1}{2} (-2 \bar \Sigma^{-1}\bar x + 2\Sigma^{-1}\bar\mu)$ \\

$\hspace{2.9cm} =  \bar \Sigma^{-1}\bar x + \Sigma^{-1}\bar\mu$ \\

since, 

$\hspace{1cm} $$\frac{\partial}{\partial\bar a} \bar b^{T}\bar a = \bar b$\\

$\hspace{0.8cm} $ $\frac{\partial}{\partial\bar a} \bar a^{T} \bar b \bar a = 2\bar b \bar a  $\\

Differentiating $\frac{\partial}{\partial \mu} \ln P_{\bar X}(\bar x)$ with respect to $\mu$\\

$ \hspace{1cm}\frac{\partial^{2}}{\partial \mu^{2}} \ln P_{\bar X}(\bar x) = -\Sigma^{-1}$\\

$\hspace{1cm}\Rightarrow -E[\frac{\partial^{2}}{\partial \mu^{2}} \ln P_{\bar X}(\bar x)] = \Sigma^{-1}$\\

Therefore, 

$\hspace{1.5cm}$ var$(\bar X) \geq \frac{1}{n\Sigma^{-1}}$\\

$\hspace{2.8cm} \geq \frac{\Sigma}{n}$\\\\

\begin{flushleft}
9) Consider the noisy signal model
\end{flushleft}

$\hspace{2cm} $ Y = X + N $\hspace{1cm}$where $\;$ N$\sim CN(0,\sigma^{2})$\\

$\hspace{2cm}$ X = A$\exp(j\phi)$

where $\hspace{1cm}$ A : amplitude

$\hspace{2cm}$ $\phi$ : random phase uniformly distributed in [0,2$\pi$]\\

Consider the real system model i.e.,

$\hspace{2cm}$ $Y_{1} = A\cos\phi + n_{1}$

$\hspace{2cm}$ $Y_{2} = A\sin\phi + n_{2} \hspace{1cm} (n_{1},n_{2} \sim N(0,\sigma^{2}))$\\

The parameter in this case can be taken as $\Theta = (\Theta_{1},\Theta_{2})$ with $\Theta_{1} \;\epsilon \;\{0,A\}$ 

and $\Theta_{2}\; \epsilon\; [0,2\pi]$.\\

The density y can be written as 

$\hspace{2.5cm} P_{\Theta}(Y) = \frac{1}{2\pi \sigma^{2}} \exp(-\frac{q(y,\theta)}{2\sigma^{2}})$\\

where $\hspace{1cm}q(y,\theta) = [(y_{1} - \theta_{1}\cos\theta_{2})^{2} + (y_{2} - \theta_{1}\sin\theta_{2})^{2}]$\\

The maximum likelihood estimation is\\

$\hspace{1cm} P(Y\mid \theta_{1}) = \frac{1}{2\pi} \int_{0}^{2\pi} P_{\Theta}(Y) \arrowvert_{\theta = A}\;d\theta_{2}$\\

$\hspace{2.5cm} = \frac{1}{4\pi^{2} \sigma^{2}} \int_{0}^{2\pi} \exp\{-\frac{q(y,\theta)\arrowvert_{\theta_{1}=A}}{2\sigma^{2}}\} \;d\theta_{2}$\\

$\hspace{2.5cm} = \frac{1}{4\pi^{2} \sigma^{2}} \int_{0}^{2\pi} \exp\{-\frac{(y_{1} - A\cos\theta_{2})^{2} + (y_{2} - A\sin\theta_{2})^{2}}{2\sigma^{2}}\} \;d\theta_{2}$\\

$\hspace{2.5cm} = \frac{1}{4\pi^{2} \sigma^{2}} \int_{0}^{2\pi} \exp\{-\frac{1}{2\sigma^{2}}(y_{1}^{2} + y_{2}^{2} + A^{2})\} \exp\{\frac{A}{\sigma^{2}}(y_{1}\cos\theta_{2} + y_{2}\sin \theta_{2})\} \;d\theta_{2}$\\

$\hspace{2.5cm} = \frac{1}{2\pi} \frac{\exp\{-\frac{1}{2\sigma^{2}}(y_{1}^{2} + y_{2}^{2} + A^{2})\}}{2\pi \sigma^{2}} \int_{0}^{2\pi} \exp\{\frac{A}{\sigma^{2}}(y_{1}\cos\theta_{2} + y_{2}\sin \theta_{2})\} \;d\theta_{2} $\\

$\hspace{2.5cm} = \frac{1}{2\pi} \frac{\exp\{-\frac{1}{2\sigma^{2}}(y_{1}^{2} + y_{2}^{2} + A^{2})\}}{2\pi \sigma^{2}}  \int_{0}^{2\pi}\exp \{\frac{Ar}{\sigma^{2}}\cos(\theta_{2} - \phi)\} \;d\theta_{2} $\\

$\hspace{2.5cm} = \frac{1}{2\pi} \frac{\exp\{-\frac{1}{2\sigma^{2}}(y_{1}^{2} + y_{2}^{2} + A^{2})\}}{2\pi \sigma^{2}} I_0(\frac{Ar}{\sigma^{2}})$\\

where $\hspace{1cm} r = \sqrt{y_{1}^{2} + y_{2}^{2}}$\\

$\hspace{2cm} \phi  = Tan^{-1}(\frac{y_{2}}{y_{1}})$\\

$\hspace{2cm} y_{1} = r\cos \phi$

$\hspace{2cm} y_{2} = r\sin\phi$\\

Bessel function of zero'th order,\\

$\hspace{2cm} I_{0}(x) = \frac{1}{2\pi} \int_{0}^{2\pi} \exp[x\cos \tau] \;d\tau $\\

Using approximation $I_{0}(x) \simeq \frac{e^{x}}{\sqrt{2\pi x}}$\\

$ \hspace{2cm}P(Y\mid \theta_{1}) = \frac{1}{2\pi} \frac{\exp\{-\frac{1}{2\sigma^{2}}(y_{1}^{2} + y_{2}^{2} + A^{2})\}}{2\pi \sigma^{2}}  \frac{e^{\frac{Ar}{\sigma^{2}}}}{\sqrt{2\pi(\frac{Ar}{\sigma^{2}})}}$\\

Differentiate $P(Y\mid \theta_{1})$ with respect to A and solve for A by equating it to 

zero.

The maximum likelihood estimation of A is $\frac{r+\sqrt{r^{2} - 2\sigma^{2}}}{2}$

\end{document}
