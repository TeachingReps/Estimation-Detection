\documentclass[12pt]{report}
\usepackage{%
	amsfonts,%
	amsmath,%	
	amssymb,%
	amsthm,%
	algorithm,%
%	babel,%
	bbm,%
	etex,%
	%biblatex,%
	caption,%
	centernot,%
	color,%
	dsfont,%
	enumerate,%
	epsfig,%
	epstopdf,%
	geometry,%
	graphicx,%
	hyperref,%
	latexsym,%
	mathtools,%
	multicol,%
	pgf,%
	pgfplots,%
	pgfplotstable,%
	pgfpages,%
	proof,%
	psfrag,%
	subfigure,%	
	tikz,%
	ulem,%
	url%
}
%\usepackage{qtree}
\usepackage[fleqn]{amsmath}
\usepackage{amsthm}
\usepackage{csquotes} 
\usepackage[noend]{algpseudocode}
\usepackage[mathscr]{eucal}
\usepgflibrary{shapes}
\usetikzlibrary{%
  	arrows,%
	backgrounds,%
	chains,%
	decorations.pathmorphing,% /pgf/decoration/random steps | erste Graphik
	decorations.text,%
	matrix,%
  	positioning,% wg. " of "
  	fit,%
	patterns,%
  	petri,%
	plotmarks,%
  	scopes,%
	shadows,%
  	shapes.misc,% wg. rounded rectangle
  	shapes.arrows,%
	shapes.callouts,%
  	shapes%
}

\theoremstyle{plain}
\newtheorem{thm}{Theorem}[section]
\newtheorem{lem}[thm]{Lemma}
\newtheorem{prop}[thm]{Proposition}
\newtheorem{cor}[thm]{Corollary}

\theoremstyle{definition}
\newtheorem{defn}[thm]{Definition}
\newtheorem{conj}[thm]{Conjecture}
\newtheorem{exmp}[thm]{Example}
\newtheorem{assum}[thm]{Assumption}
\newtheorem{axiom}[thm]{Axiom}
\newtheorem{theorem}{Theorem}[section]
\newtheorem{corollary}{Corollary}[theorem]
\newtheorem{lemma}[theorem]{Lemma}



\theoremstyle{remark}
\newtheorem{rem}{Remark}
\newtheorem{note}{Note}
\newtheorem{fact}{Fact}

\newcommand{\norm}[1]{\left\lVert#1\right\rVert}
\newcommand{\indep}{\!\perp\!\!\!\perp}
\DeclarePairedDelimiter\abs{\lvert}{\rvert}%
\newcommand\numberthis{\addtocounter{equation}{1}\tag{\theequation}}
\newcommand{\tr}{\operatorname{tr}}
\newcommand{\R}{\mathbb{R}}
\newcommand{\N}{\mathbb{N}}
\newcommand{\E}{\mathbb{E}}
\newcommand{\Z}{\mathbb{Z}}
\newcommand{\B}{\mathscr{B}}
\newcommand{\C}{\mathcal{C}}
\newcommand{\T}{\mathscr{T}}
\newcommand{\F}{\mathcal{F}}
\newcommand{\G}{\mathcal{G}}
%\newcommand{\ba}{\begin{align*}}
%\newcommand{\ea}{\end{align*}}
\newcommand{\expect}[1]{\mathbb{E}\left[{#1}\right]}
\newcommand{\prob}[1]{\mathbb{P}\left[{#1}\right]}
\newcommand{\probo}[1]{\mathbb{P}_0\left[{#1}\right]}
\newcommand{\probi}[1]{\mathbb{P}_1\left[{#1}\right]}
\newcommand{\given}{\; \big\vert \;} 
\newcommand{\bydef}{:=}
\newcommand{\indic}[1]{\mathbbm{1}\{#1\}}
\DeclareMathOperator*{\argmax}{arg\,max}
\renewcommand{\qedsymbol}{$\blacksquare$}
\makeatletter
\def\BState{\State\hskip-\ALG@thistlm}
\makeatother

\makeatletter
\def\th@plain{%
  \thm@notefont{}% same as heading font
  \itshape % body font
}
\def\th@definition{%
  \thm@notefont{}% same as heading font
  \normalfont % body font
}
\makeatother
\date{}
\usepackage{scribe_e1244}
\usepackage{times}
\begin{document}
\lecturer{Aditya Gopalan}	% optional, put lecturer's name here
\scribe{Namrata Kar, Nanditha Unnikrishnan}		% required, put your name here
\lecturenumber{16}			% required, must be a number
\lecturedate{February 28}		% required, omit year
\maketitle

% title of the lecture
\begin{center}
{\Large \bf Sequential Detection}
\end{center}

% ----------------------------------------------------------------------


\section{Recap}
\subsection{The Optimal Bayes' Rule :}
The optimal rule is:
\begin{eqnarray}
\label{l10}
\phi_n(y_1,y_2,\cdots,y_n) =  
\begin{cases}
0 \ \ \ \ if \ \pi_1(y_1,y_2,\cdots,y_n)\in[\pi_L,\pi_u] \\
1 \ \ \ \ if \ otherwise
\end{cases}
\end{eqnarray}

\begin{eqnarray}
\label{l10}
\delta_n(y_1,y_2,\cdots,y_n) =  
\begin{cases}
0 \ \ \ \ if \ \pi_1(y_1,y_2,\cdots,y_n) < \pi_L \\
1 \ \ \ \ if \ \pi_1(y_1,y_2,\cdots,y_n) > \pi_u \\
0 \ or \ 1 \ \ if \ otherwise\ (doesn't \ matter!)
\end{cases}
\end{eqnarray}
We can rewrite the optimal Bayes rule as:\newline

\begin{eqnarray}
%\label{l10}
\phi_n(y_1,y_2,\cdots,y_n) =  
\begin{cases}
0 \ \ \ \ if \ L_n\in [\underline{\pi},\overline{\pi}] \\
1 \ \ \ \ otherwise\\
\end{cases}
\end{eqnarray}

\begin{eqnarray}
%\label{l10}
\delta_n(y_1,y_2,\cdots,y_n) =  
\begin{cases}
0 \ \ \ \ if \ L_n < \underline{\pi} \\
1 \ \ \ \ if \ L_n > \overline{\pi} \\
0 \ or \ 1 \ \ if \ otherwise\ (doesn't \ matter!)
\end{cases}
\end{eqnarray}

where\newline
\begin{eqnarray}
\underline{\pi}=\frac{\pi_0\pi_L}{\pi_1(1-\pi_L)}\\
\overline{\pi}=\frac{\pi_0\pi_u}{\pi_1(1-\pi_u)}
\end{eqnarray}

\section{Definition: SPRT(a,b)}

A sequential Probability Ratio Test with parametes a$ \leq$ b is the rule:
\newline
\newline
SPRT (a,b) : Sample until the first time L$_n$( $y_1,y_2,y_3,.....y_n )$ falls outside (a,b).
With a = $\underline{\pi}$ and b = $\overline{\pi}$, SPRT (a,b) becomes Bayes optimal test. Choose H$_1$ if L$_n$($y_1,y_2,y_3,.....y_n)$ $>$ b and H$_0$ if L$_n$($y_1,y_2,y_3,.....y_n)$ $<$ a.\\
\newline
\textbf{Note:}
\newline
L$_0$ $(.)$ is defined to be equal to one. If either a$ > $1 or b$ < $1 , then the resulting SPRT takes no samples.
\newline
\\\textbf{Example:}
\begin{align*}
\\H_0 : Y_k = N_k \ \  \ \ \ \ \ k = 1,2,3\dots
\\H_1 : Y_k = $ \theta$ + \textit{N}_k \ \ k = 1,2,3\dots
\end{align*}
where $\theta$ $\in$ \Re $^{+}$ and ${$\textit{N}_k$}\stackrel{iid}\sim $ \textit{N}$(0,\sigma^{2} )$
\newline
\begin{eqnarray}
L$_n$($y_1,y_2,y_3,\cdots,y_n)$&=&\prod^{n}_{k=1} \frac{P_1(y_k)}{P_0(y_k)}\\
&=&\prod^{n}_{k=1} \frac{\frac{1}{\sigma\sqrt{2\pi}}e^{\frac{-(y_k-\theta)^{2}}{2\sigma^{2}}}}{\frac{1}{\sigma\sqrt{2\pi}}e^{\frac{-(y_k)^{2}}{2\sigma^{2}}}}\\
&=&exp \lbrace \ \  \sum^{n}_{k=1} \theta (y_k - \frac{\theta}{2} ) / \sigma^{2} \ \  \rbrace
\end{eqnarray}


SPRT(a,b) :  Sample until $\frac{\theta}{\sigma^{2}}$ $ \sum^{n}_{k=1} ( y_k -\frac{\theta}{2} )$ does not belong to ( log(a) , log(b) ).

\section{Sample Optimality of the Bayes' SPRT ( SPRT( $\underline{\pi}$,$\overline{\pi}$) )}

For a sequential rule ($\underline{\phi}$,$\underline{\delta}$),
\newline
\begin{eqnarray}
P_{F}(\underline{\phi},\underline{\delta})&:=&\textit{P}_{0}[\delta_{N}(Y_1,Y_2,Y_3,...,Y_N)]\\
P_{M}(\underline{\phi},\underline{\delta}) &:=& \textit{P}_{1}[1 - \delta_{N}(Y_1,Y_2,Y_3,..,Y_N)]\\
N(\underline{\phi})&:=& min\lbrace n \geq 0 : \phi_{n}(Y_1,Y_2,Y_3,...,Y_n) = 1\rbrace
\end{eqnarray}
\section{Wald-Wolfowitz Theorem : [ $\sim$ 1943-1950]}
Suppose ($\underline{\phi}^{*}$,$\underline{\delta}^{*}$) represents the Bayes' optimal SPRT($\underline{\pi}$,$\overline{\pi}$), and that ($\underline{\phi}$,$\underline{\delta}$) is any sequential rule, such that 
\begin{eqnarray}
P_{F}(\underline{\phi},\underline{\delta}) \leq P_{F}(\underline{\phi}^{*},\underline{\delta}^{*}),\\
P_{M}(\underline{\phi},\underline{\delta}) \leq P_{M}(\underline{\phi}^{*},\underline{\delta}^{*})
\end{eqnarray}
Then,
\begin{equation}
\textit{E}_{j} [N(\underline{\phi})] \geq \textit{E}_{j} [N(\underline{\phi^{*}})] \ \ \forall \ j\in \lbrace 0 , 1\rbrace
\end{equation}
\newline
[Proof: Coming up in Homework]
\newline
\newline
\subsection{Wald's Approximation}
\textbf{Question:} How can we set (a,b) in SPRT(a,b) to achieve desired performace ?\\
\newline
Say, P$_{F}$ \leq $\alpha$ , P$_{M}$ \leq $\gamma$ and suppose ($\underline{\phi}$,$\underline{\delta}$) is the SPRT(a,b).\\
The false alarm probability of ($\underline{\phi}$,$\underline{\delta}$)
\begin{eqnarray}
\alpha &=& P_{F}(\underline{\phi},\underline{\delta})\\
&=&\textit{P}_{0}[\Gamma_{1}]
\end{eqnarray}
where $\Gamma_{1}$ is the "rejection region" of ($\underline{\phi}$,$\underline{\delta}$).
\begin{equation}
\Gamma_{1} &:=&\lbrace (y_1,y_2,....) : L_{N}(y_1,y_2,...y_N) > b\rbrace
\end{equation}
\begin{eqnarray}
\alpha &=&\textit{P}_{0}[\Gamma_{1}]\\
&=&\textit{P}_{0} [ \cup_{n=0}^{\infty} \lbrace (y_1,y_2,....) : N = n, L_{n}(y_1,y_2,...y_n) > b\rbrace ]\\
&=&\sum_{n=0}^{\infty}\textit{P}_{0}[Q_{n}]
\end{eqnarray}
where Q_{n} $=$ $\lbrace$ (y_1,y_2,....) : N = n, L_{n}(y_1,y_2,...y_n) > b $\rbrace$
\newline
\begin{eqnarray}
\alpha &=&\sum_{n=0}^{\infty}\textit{P}_{0}[Q_{n}]\\
&=& \sum_{n=0}^{\infty}\int_{Q_{n}} P_{0}(\underline{y}) d\underline{y}\\
\end{eqnarray}
As $\underline{y}$ $=$ $(y_1,y_2,....)$ $\in$ $Q_{n}$, P$_{0}$($\underline{y}$) is dependent only on the first n samples of $\underline{y}$.
\begin{eqnarray}
\alpha &=& \sum_{n=0}^{\infty}\int_{Q_{n}} \prod_{k=1}^{n} P_{0} (y_k) dy_k\\
&\leq & \sum_{n=0}^{\infty} \frac{1}{b} \int_{Q_{n}} \prod_{k=1}^{n} P_{1} (y_k) dy_k
\end{eqnarray}
\newline
Since $\underline{y}$ $\in$ Q$_{n}$, L$_{n}$($y_1,y_2,...y_n$) $>$ b , \\
\begin{equation}
\\\prod_{k=1}^{n} P_{1}(y_k) \geq b( \prod_{k=1}^{n} P_{0}(y_k) )
\end{equation}
\begin{eqnarray}
\alpha &\leq & \sum_{n=0}^{\infty} \frac{1}{b} \int_{Q_{n}} \prod_{k=1}^{n} P_{1} (y_k) dy_k\\
&=& \frac{1}{b} \textit{P}_{1}[ L_{N}(y_1,y_2,...y_N) > b]\\
&=& \frac{1}{b} (1 - \gamma)
\end{eqnarray}
\newline
Similarly, $\gamma$ $\leq$ a ( 1 - $\alpha$ ) (Following the same arguement as above for P$_M$).\\
So we have,
\begin{eqnarray}
b &\leq & \frac{1-\gamma}{\alpha}\\
a &\geq & \frac{\gamma}{1-\alpha}
\end{eqnarray}
\newline
Hence any choice of (a,b) satisfying the above inequalities will guarantee P$_{F}$ $\leq$ $\alpha$ and P$_{M}$\leq $\gamma$.
\newline
\section{Wald's Lemma}
Let Z$_1$, Z$_2$ ,.... be an i.i.d sequence of random variables, with \textit{E}[Z$_1$] $=$ 
$\mu$.
Consider K, a non-negative integer-valued random variable such that
\begin{itemize}
	\item \textit{E}[K] $<$ $\infty$ ,
	\item $\forall$ k integers ,the event $\lbrace$ K = k$\rbrace$ is completely determined by Z$_1$, Z$_2$ ,....Z$_k$.
\end{itemize}
Then, 
\begin{eqnarray}
\textit{E}[\sum_{k=1}^{K} Z_k ] &=& \mu \textit{E}[K] \\
N &=& min\lbrace n \geq 0 : \phi_{n} ( Y_1,Y_2,....Y_n ) = 1 \rbrace
\end{eqnarray}
\newline
$\forall$ k , $\lbrace$ N $=$ k $\rbrace$ is completely determined by (Y$_1$, Y$_2$,....Y$_k$)
\newline
\subsection{Calculation of Expected number of samples in a SPRT}
\textbf{Question:}
What is the expected sample number of SPRT(a,b) under each hypothesis ?
\newline
Let us assume that \textit{E}$_{j}$ [N] $<$ $\infty$.
\newline
\begin{eqnarray}
\textit{E}_{0}[ log L_{N} (Y_1, Y_2,...,Y_N) ] &=& \textit{E}_{0}[ \sum_{i=1}^{N} log \frac{P_{1}(Y_i)}{P_{0}(Y_i)}]\\
&=& - \textit{E}_{0}[ \sum_{i=1}^{N} log \frac{P_{0}(Y_i)}{P_{1}(Y_i) }]\\
&=& - \textit{E}_{0}[N] \  \textit{E}_{0}[log \frac{P_{0}(Y_1)}{P_{1}(Y_1) }]
\end{eqnarray} 
(The above equation can be obtained using Wald's Lemma.)
\newline
where,
\begin{equation}
{E}_{0}[log \frac{P_{0}(Y_1)}{P_{1}(Y_1) }] = D(P_{0} \Vert P_{1}) 
\end{equation}
\newline
is the Kullback-Leibler Divergence between P$_0$ and P$_1$.
\newline
\newline
Similarly, 
\begin{equation}
\textit{E}_{1}[ log L_{N} (Y_1, Y_2,...,Y_N) ] = \textit{E}_{1}[N] D(P_{1} \Vert P_{0})
\end{equation}
\newline
\newline
Alternatively, assume that $:$
\begin{equation}
L_{N} (Y_1, Y_2,...,Y_N) \approx b \  or \ L_{N} (Y_1, Y_2,...,Y_N) \approx a
\end{equation}
\newline
(overshoot or undershoot $\approx$ 0).
\begin{eqnarray}
\textit{E}_{0}[ log L_{N} (Y_1, Y_2,...,Y_N) ] &\approx & ( log a ) ( 1 - P_{F} ) + ( log b ) ( P_{F} )\\
\textit{E}_{1}[ log L_{N} (Y_1, Y_2,...,Y_N) ] &\approx & ( log a ) ( P_{M} ) + ( log b ) ( 1 - P_{M} )
\end{eqnarray}
\newline
\newline
Using equations (16.37) ,(16.39) ,(16.41) and (16.42) \\
\begin{eqnarray}
\textit{E}_{0}[ N ] &\approx & \frac{(1 - \alpha ) log \frac{(1 - \alpha )}{\gamma} + \alpha log \frac{\alpha}{( 1 - \gamma )} }{D(P_{0} \Vert P_{1})}\\
\textit{E}_{1}[ N ] &\approx & \frac{(1 - \gamma ) log \frac{(1 - \gamma )}{\alpha} + \gamma log \frac{\gamma}{( 1 - \alpha )} }{D(P_{1} \Vert P_{0})}
\end{eqnarray}
\newline
The closer the distributions P$_0$ and P$_1$ are, the harder it becomes to distiguish between them and hence we need more number of samples (which is evident from the above equations).
\bibliography{Lecture_016_Scribe_Notes}
\bibliographystyle{IEEEtran}
\end{document}